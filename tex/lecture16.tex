\chapter{Categorical Semantics}
\lecturedetails{28 November 2017}{M Fiore, A Hammond, S Steenkamp}

\newcommand{\nat}{\textbf{nat}}
\newcommand{\bool}{\textbf{bool}}
\newcommand{\ctxt}{\text{Ctxt}}
\newcommand{\eqncode}[1]{\mathbin{\text{\textbf{#1}}}}

Type theories extend algebraic theories, and the relation between them can be
made mathematically precise. We can extend even further to get dependent type
theory.
\begin{align*}
    \text{\textbf{Algebraic theories}}\quad &\hookrightarrow\quad
    \text{\textbf{Type theories}} \\
    \text{sorts}\quad &\hookrightarrow\quad \text{types} \\
    \text{operators}\quad &\leadsto\quad \text{operators} \\
    \text{terms}\quad &\mathrel{\phantom{\hookrightarrow}}\quad \text{variable
    binding} \\
    \text{equational logic}\quad &
\end{align*}
Sorts have no structure, just names, whereas types have structure.

\section{Sorts (a.k.a. basic types)}

Consider a simple language with some basic types $\beta$ built in (e.g. \nat,
\bool), which may also have some operators (e.g. $+$, $\&\&$, etc.). Then we can
define contexts which give types to variables
\begin{align*}
    \Gamma = (x_1 : \beta_1,\ \ldots,\ x_n : \beta_n)
\end{align*}
The context can be thought of as a list of variables with associated types. If
each variable is numbered (var1, var2, \ldots), then the context can just be a
list of types $(\beta_1,\ \beta_2,\ \ldots)$, with the position in the list
indicating which variable the type is associated to.

The rules for the context are:
\begin{align*}
    \infer[\text{empty context}]{\bullet : \ctxt}{} &&
    \infer[\text{extra ``assumption''}]
    {(\Gamma,\ x : \beta) : \ctxt}{\Gamma : \ctxt}
\end{align*}
The only terms that are valid are those variables that are in the context:
\begin{align*}
    x_1 : \beta_1,\ \ldots,\ x_n : \beta_n \quad \vdash \quad x_i : \beta_i
    && \text{(for $i = 1, \ldots, n$)}
\end{align*}
since the only rule for typing is
\begin{align*}
    \infer[(i \in (1, \ldots, n))]{x_1 : \beta_1,\ \ldots,\ x_n :
    \beta_n\ \vdash\ x_i : \beta_i}{}
\end{align*}

\section{Category of Contexts}

The objects are contexts (i.e. lists of [variables with associated] types).

The morphisms from $\Gamma$ to $\Delta$ are $m$-tuples of terms in context
$\Gamma$ of type $\sigma_i$, where the length of $\Delta$ is $m$ and it's
elements are $\sigma_i$ for $i \in (1, \ldots, m)$. That is, if $\Gamma = (x_1 :
\beta_1,\ \ldots,\ x_n : \beta_n)$ and $\Delta = (y_1 : \sigma_1,\ \ldots,\ y_m
: \sigma_m)$, then a morphism $\Gamma \rightarrow \Delta$ is an $m$-tuple
$(t_1,\ \ldots, t_m)$ of terms $t_i$ where each term $t_i$ has type $\sigma_i$
under the context $\Gamma$, i.e., $\Gamma \vdash t_i : \sigma_i$.

Identities $\Gamma \rightarrow \Gamma$ are just the list of variables
$(x_1,\ \ldots, x_n)$.

Composition is given by substitution:
\begin{center}
\begin{tikzcd}
    \Gamma \arrow[rr, "{(\Gamma \vdash t_i : \tau_i)_{i = 1,\ldots,p}}"] &  &
    \Delta \arrow[rr, "{(\Delta \vdash u_j : \sigma_j)_{j = 1, \ldots, q}}"] &
    & \Theta
\end{tikzcd}
\end{center}
\begin{align*}
    &(\Delta \vdash u_j : \sigma_j)_{j = 1,\ldots,q} \circ
    (\Gamma \vdash t_i :   \tau_i)_{i = 1,\ldots,p} \\
    =\ \ &(\Gamma \vdash u_j[\sfrac{t_1}{x_1},\ \ldots,\ \sfrac{t_n}{x_n}] :
    \sigma_j)_{j = 1,\ldots,q}
\end{align*}
This seems very trivial, but it is not completely trivial.

The category of contexts is the free Cartesian category on the set of base
types.
\begin{center}
\begin{tikzcd}
    &  & \Gamma = ((x_1, \tau_1), (x_2, \tau_2), \ldots, (x_n, \tau_n)
    \arrow[lldd, "(\Gamma \vdash x_1 : \tau_1)"'] \arrow[ldd, "(\Gamma \vdash
    x_2 : \tau_2)"] \arrow[rrdd, "(\Gamma \vdash x_n : \tau_n)"] &  &  \\
    &  & \dots &  &  \\
    (x_1, \tau_1) & (x_2, \tau_2) & \dots &  & (x_n, \tau_n)
\end{tikzcd}
\end{center}
(Prove that this is the product.)

New let us gradually build up the type system, first we will add product types.

\section{Product types}

The types are extended to
\begin{align*}
    \tau \Coloneqq \beta \mid \text{unit} \mid \tau_1 \ast \tau_2
\end{align*}
with typing rules
\begin{align*}
    \infer{\Gamma \vdash (t_1, t_2) : \tau_1 \ast \tau_2}{\Gamma \vdash t_1 :
    \tau_1 & \Gamma \vdash t_2 : \tau_2}
\end{align*}
\begin{align*}
    \infer{\Gamma \vdash \text{fst}(t) : \tau_1}{\Gamma \vdash t : \tau_1 \ast
    \tau_2}
    &&
    \infer{\Gamma \vdash \text{snd}(t) : \tau_2}{\Gamma \vdash t : \tau_1 \ast
    \tau_2}
\end{align*}

We would like to give meaning to these symbols.

\section{Categorical semantics in Cartesian categories}

\subsection{For product types}

Given a Cartesian category $\mathcal{C}$ with $\llbracket \beta \rrbracket \in
\mathbb{C}$.

Interpret contexts $\Gamma = (x_1 : \tau_1,\ \ldots,\ x_n :
\tau_n)$ as $\llbracket \Gamma \rrbracket = \llbracket \tau_1 \rrbracket \times
\ldots \times \llbracket \tau_n \rrbracket$,

the unit type $()$ as $\llbracket () \rrbracket = 1$,

and product types $\tau_1 \ast \tau_2$ as $\llbracket \tau_1 \ast \tau_2
\rrbracket = \llbracket \tau_1 \rrbracket \times \llbracket \tau_2 \rrbracket$.

Now we ask the question: What equational theory need be imposed on the type
theory so that the interpretation is sound and complete?

In $\mathbb{C}$ the typing inference
\begin{tikzcd}
    \llbracket \Gamma \rrbracket \arrow[rr, "\llbracket \Gamma \vdash t : \tau
    \rrbracket"] \arrow[rr, "\text{(projection)}"'] &  & \llbracket \tau
    \rrbracket
\end{tikzcd}
is given by projection.

Specifically, we define:
\begin{align*}
    \llbracket \Gamma \vdash x_i : \tau_i \rrbracket &\eqdef
    \pi_i : \llbracket \tau_1 \rrbracket \times \ldots \times \llbracket \tau_n
    \rrbracket \rightarrow \llbracket \tau_i \rrbracket
    \\
    \llbracket \Gamma \vdash (t_1, t_2) : \tau_1 \ast \tau_2 \rrbracket &\eqdef
    \big(\,\llbracket \Gamma \vdash t_1 : \tau_1 \rrbracket,\,\,
           \llbracket \Gamma \vdash t_2 : \tau_2 \rrbracket\,\big)
    \\
    \llbracket \Gamma \vdash \text{fst}(t) : \tau_1 \rrbracket &\eqdef
    \pi_1 \circ \llbracket \Gamma \vdash t : \tau_1 \ast \tau_2 \rrbracket
    \\
    \llbracket \Gamma \vdash \text{snd}(t) : \tau_2 \rrbracket &\eqdef
    \pi_2 \circ \llbracket \Gamma \vdash t : \tau_1 \ast \tau_2 \rrbracket
\end{align*}

\subsection{What does soundness and completeness mean?}

Soundness means that
\begin{align*}
    \text{if}\quad& \Gamma \vdash t = u : \tau \implies \llbracket \Gamma \vdash t
    : \tau \rrbracket
    \\
    \text{and}\quad& \Gamma \vdash t = u : \tau \implies \llbracket \Gamma \vdash u
    : \tau \rrbracket
    \\
    \text{then}\quad& \llbracket \Gamma \vdash t : \tau \rrbracket = \llbracket
    \Gamma \vdash u : \tau \rrbracket
\end{align*}
and completeness means
\begin{align*}
    &\forall \Gamma \vdash t : \tau, \forall \Gamma \vdash u : \tau, \forall
    \text{Cartesian closed categories } \mathbb{C},
    \\
    &\mathbb{C}\llbracket \Gamma \vdash t : \tau \rrbracket =
    \mathbb{C}\llbracket \Gamma \vdash u : \tau \rrbracket \implies
    \Gamma \vdash t = u : \tau
\end{align*}

\subsection{Equational theroy for product types}

So, what equational theory do we need to impose on product types so that the
interpretation is sound and complete?
\begin{align*}
    \text{fst}((t_1, t_2)) =\; &t_1 : \tau_1
    \\
    \text{snd}((t_1, t_2)) =\; &t_2 : \tau_2
    \\
    t = (\text{fst}(t), \text{snd}(t)) :\; &\tau_1 \ast \tau_2
\end{align*}

Note that it is not always possible to find a (finite) set of rules.

Now we will look at (the categorical semantics for) sum types and function types.

\section{Sum types}

The types are extended to
\begin{align*}
    \tau \Coloneqq \ldots \mid \varnothing \mid \tau_1 + \tau_2
\end{align*}
with the following typing rules
\begin{align*}
    \infer{\Gamma \vdash (\eqncode{case}\ t \eqncode{of} x_1 : \tau_1 \Rightarrow
    t_1 \mid x_2 : \tau_2 \Rightarrow t_2) : \tau}
    {\Gamma \vdash t : \tau_1 + \tau_2 & \Gamma, x_1 : \tau_1 \vdash t_1 : \tau
    & \Gamma, x_2 : \tau_2 \vdash t_2 : \tau}
\end{align*}
\begin{align*}
    \infer{\Gamma \vdash \eqncode{inl}(t_1) : \tau_1 + \tau_2}
    {\Gamma \vdash t_1 : \tau_1}
    &&
    \infer{\Gamma \vdash \eqncode{inr}(t_2) : \tau_1 + \tau_2}
    {\Gamma \vdash t_2 : \tau_2}
\end{align*}
\begin{align*}
    \infer{\Gamma \vdash \bot_\tau(t) : \tau}
    {\Gamma \vdash t : \varnothing}
\end{align*}
This last rule says that if you can get something in the empty type then you can
get something in any type. In fact this rule is suggested by the mathematics:
\begin{center}
\begin{tikzcd}
    \llbracket \Gamma \rrbracket \arrow[rr, "\llbracket \Gamma \vdash
    \bot_\tau(t) : \tau \rrbracket"] \arrow[rd, "\llbracket \Gamma \vdash t :
    \varnothing \rrbracket"'] &  & \llbracket \tau \rrbracket \\
    & 0 \arrow[ru, "\exists!0_\tau"'] &
\end{tikzcd}
\end{center}
The intepretations of $\varnothing$ and $\tau_1 + \tau_2$ are:
\begin{align*}
    &\llbracket \varnothing \rrbracket = 0
    \\
    &\llbracket \tau_1 + \tau_2 \rrbracket = \llbracket \tau_1 \rrbracket +
    \llbracket \tau_2 \rrbracket
\end{align*}
Additionally in a category with coproducts we have
\begin{center}
\begin{tikzcd}
    \llbracket t_1 \rrbracket \arrow[rrd, "\iota_1"] &  &  \\
    &  & \llbracket t_1 \rrbracket + \llbracket t_2 \rrbracket \\
    \llbracket t_2 \rrbracket \arrow[rru, "\iota_2"'] &  &
\end{tikzcd}
\end{center}

\begin{exercise}
    Give a categorial interpretation of sum types in a distributive category,
    crucially using $\delta$ (it is not possible in a non-distributive
    category).
\end{exercise}

The equational theory that must be satisfied is:
\begin{align*}
    \eqncode{case}(\eqncode{inl}(t)) \eqncode{of} x_1 \Rightarrow t_1 &\mid x_2
    \Rightarrow t_2\quad =\quad t_1[\sfrac{t}{x_1}]
    \\
    \eqncode{case}(\eqncode{inr}(t)) \eqncode{of} x_1 \Rightarrow t_1 &\mid x_2
    \Rightarrow t_2\quad =\quad t_2[\sfrac{t}{x_2}]
    \\
    e = \eqncode{case}\ e \eqncode{of} x \Rightarrow \eqncode{inl}(x) &\mid x
    \Rightarrow \eqncode{inr}(x)
\end{align*}
Although to show this works we need to consider $\llbracket t_1[\sfrac{t}{x_1}]
\rrbracket$, but what is the (categorical) interpretation of substitution?

The substitution rule is
\begin{align*}
    \infer[\text{(sub)}]
    {\Gamma \vdash t[\sfrac{u}{x}] : \sigma}
    {\Gamma, x : \tau \vdash t : \sigma & \Gamma \vdash u : \tau}
\end{align*}
and the interpretation is
\begin{center}
\begin{tikzcd}
    \llbracket \Gamma \rrbracket \arrow[rr, "{\llbracket \Gamma \vdash
    t[\sfrac{u}{x}] : \sigma \rrbracket}"] \arrow[rdd, "{\langle \text{id},
    \llbracket u \rrbracket \rangle}"'] &  & \llbracket \sigma \rrbracket \\
    &  &  \\
    & \llbracket \Gamma \rrbracket \times \llbracket \tau \rrbracket \arrow[ruu,
    "\llbracket t \rrbracket"'] &
\end{tikzcd}
\end{center}
this commutes by the \emph{substitution by composition} lemma.

\section{Function Types}

The types can also be extended with 
\begin{align*}
    \tau \Coloneqq \ldots \mid \tau_1\rightarrow\tau_2
\end{align*}
with the following typing rules
\begin{align*}
    \infer{\Gamma\vdash t(t_1) : \tau_2}
    {\Gamma\vdash t : \tau_1 \rightarrow \tau_2 & \Gamma\vdash t : \tau_1}
\end{align*}
\begin{align*}
    \infer{\Gamma\vdash\lambda x:\tau_1 .\; t : (\tau_1 \rightarrow \tau_2)}
    {\Gamma, x:\tau_1\vdash t : \tau_2}
\end{align*}
In a cartesian closed category the interpretation of $\tau_1 \rightarrow \tau_2$
is:
\begin{align*}
    \llbracket\tau_1 \rightarrow \tau_2\rrbracket =
    \llbracket\tau_1\rrbracket\Rightarrow\llbracket\tau_2\rrbracket
\end{align*}

The interpretation of function abstraction is:
\begin{align*}
    \llbracket\Gamma\vdash\lambda x:\tau_1 .\; t :
    \tau_1\rightarrow\tau_2\rrbracket
    \eqdef
    \llbracket\Gamma\rrbracket
    \times
    \llbracket\tau_1\rrbracket
    \rightarrow
    \llbracket\tau_2\rrbracket
\end{align*}
or equivalently
\begin{align*}
    \llbracket\Gamma\vdash\lambda x:\tau_1 .\; t :
    \tau_1\rightarrow\tau_2\rrbracket
    \eqdef
    \llbracket\Gamma\rrbracket
    \rightarrow
    \llbracket\tau_1\rrbracket
    \Rightarrow
    \llbracket\tau_2\rrbracket
\end{align*}
and the interpretation of function application is
\begin{align*}
    \llbracket\Gamma\vdash t_1(t_2) : \tau_2\rrbracket
    \eqdef
    \llbracket\Gamma\rrbracket
    \rightarrow
    (\llbracket\tau_1\llbracket \Rightarrow \llbracket\tau_2\rrbracket)
    \times
    \llbracket\tau_1\rrbracket
\end{align*}

These satisfy the equational theory
\begin{align*}
    (\lambda x.\; t) u = t[u/x] \\
    (\lambda x.\; t x) = t  && x \not\in FV(t)
\end{align*}

\section{Curry-Howard Isomorphism}

If we consider the category of contexts with all the terms removed, such that the objects are just tuples of types,

\begin{align*}
    \Gamma \equiv \tau_1,\tau_2,\tau_3...\tau_n
\end{align*}

then the typing rules become identical to logical inference rules, e.g.

\begin{align*}
    \infer{\Gamma\vdash \tau_1 \times \tau_2}
    {\Gamma\vdash\tau_1 & \Gamma\vdash\tau_2}
\end{align*}

which is the usual inference rule for logical conjunction, where the types
represent propositions. This is the Curry-Howard Isomorphism.