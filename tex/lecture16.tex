\chapter{Categorical Semantics}
\lecturedetails{28 November 2017}{M Fiore, A Hammond, S Steenkamp}

Type theories extend algebraic theories, and the relation between them can be
made mathematically precise. We can extend even further to get dependent type
theory.
\begin{align*}
    \text{\textbf{Algebraic theories}}\quad &\hookrightarrow\quad
    \text{\textbf{Type theories}} \\
    \text{sorts}\quad &\hookrightarrow\quad \text{types} \\
    \text{operators}\quad &\leadsto\quad \text{operators} \\
    \text{terms}\quad &\mathrel{\phantom{\hookrightarrow}}\quad \text{variable
    binding} \\
    \text{equational logic}\quad &
\end{align*}
Sorts have no structure, just names, whereas types have structure.

\section{Sorts (a.k.a. basic types)}

\newcommand{\nat}{\textbf{nat}}
\newcommand{\bool}{\textbf{bool}}
\newcommand{\ctxt}{\text{Ctxt}}

Consider a simple language with some basic types $\beta$ built in (e.g. \nat,
\bool), which may also have some operators (e.g. $+$, $\&\&$, etc.). Then we can
define contexts which give types to variables
\begin{align*}
    \Gamma = (x_1 : \beta_1,\ \ldots,\ x_n : \beta_n)
\end{align*}
The context can be thought of as a list of variables with associated types. If
each variable is numbered (var1, var2, \ldots), then the context can just be a
list of types $(\beta_1,\ \beta_2,\ \ldots)$, with the position in the list
indicating which variable the type is associated to.

The rules for the context are:
\begin{align*}
    \infer[\text{empty context}]{\bullet : \ctxt}{} &&
    \infer[\text{extra ``assumption''}]
    {(\Gamma,\ x : \beta) : \ctxt}{\Gamma : \ctxt}
\end{align*}
The only terms that are valid are those variables that are in the context:
\begin{align*}
    x_1 : \beta_1,\ \ldots,\ x_n : \beta_n \quad \vdash \quad x_i : \beta_i
    && \text{(for $i = 1, \ldots, n$)}
\end{align*}
since the only rule for typing is
\begin{align*}
    \infer[(i \in (1, \ldots, n))]{x_1 : \beta_1,\ \ldots,\ x_n :
    \beta_n\ \vdash\ x_i : \beta_i}{}
\end{align*}

\section{Category of Context}

The objects are contexts (i.e. lists of [variables with associated] types).

The morphisms from $\Gamma$ to $\Delta$ are $m$-tuples of terms in context
$\Gamma$ of type $\sigma_i$, where the length of $\Delta$ is $m$ and it's
elements are $\sigma_i$ for $i \in (1, \ldots, m)$. That is, if $\Gamma = (x_1 :
\beta_1,\ \ldots,\ x_n : \beta_n)$ and $\Delta = (y_1 : \sigma_1,\ \ldots,\ y_m
: \sigma_m)$, then a morphism $\Gamma \rightarrow \Delta$ is an $m$-tuple
$(t_1,\ \ldots, t_m)$ of terms $t_i$ where each term $t_i$ has type $\sigma_i$
under the context $\Gamma$, i.e., $\Gamma \vdash t_i : \sigma_i$.

Identities $\Gamma \rightarrow \Gamma$ are just the list of variables
$(x_1,\ \ldots, x_n)$.

Composition is given by substitution:
\begin{center}
\begin{tikzcd}
    \Gamma \arrow[rr, "{(\Gamma \vdash t_i : \tau_i)_{i = 1,\ldots,p}}"] &  &
    \Delta \arrow[rr, "{(\Delta \vdash u_j : \sigma_j)_{j = 1, \ldots, q}}"] &
    & \Theta
\end{tikzcd}
\end{center}
\begin{align*}
    &(\Delta \vdash u_j : \sigma_j)_{j = 1,\ldots,q} \circ
    (\Gamma \vdash t_i :   \tau_i)_{i = 1,\ldots,p} \\
    =\ \ &(\Gamma \vdash u_j[\sfrac{t_1}{x_1},\ \ldots,\ \sfrac{t_n}{x_n}] :
    \sigma_j)_{j = 1,\ldots,q}
\end{align*}
This seems very trivial, but it is not completely trivial.

The category of contexts is the free Cartesian category on the set of base
types.
\begin{center}
\begin{tikzcd}
    &  & \Gamma = ((x_1, \tau_1), (x_2, \tau_2), \ldots, (x_n, \tau_n)
    \arrow[lldd, "(\Gamma \vdash x_1 : \tau_1)"'] \arrow[ldd, "(\Gamma \vdash
    x_2 : \tau_2)"] \arrow[rrdd, "(\Gamma \vdash x_n : \tau_n)"] &  &  \\
    &  & \dots &  &  \\
    (x_1, \tau_1) & (x_2, \tau_2) & \dots &  & (x_n, \tau_n)
\end{tikzcd}
\end{center}
(Prove that this is the product.)

New let us gradually build up the type system, first we will add product types.

\section{Product types}

The types are extended to
\begin{align*}
    \tau \Coloneqq \beta \mid \text{unit} \mid \tau_1 \ast \tau_2
\end{align*}
with typing rules
\begin{align*}
    \infer{\Gamma \vdash (t_1, t_2) : \tau_1 \ast \tau_2}{\Gamma \vdash t_1 :
    \tau_1 & \Gamma \vdash t_2 : \tau_2}
\end{align*}
\begin{align*}
    \infer{\Gamma \vdash \text{fst}(t) : \tau_1}{\Gamma \vdash t : \tau_1 \ast
    \tau_2}
    &&
    \infer{\Gamma \vdash \text{snd}(t) : \tau_2}{\Gamma \vdash t : \tau_1 \ast
    \tau_2}
\end{align*}

We would like to give meaning to these symbols.

\section{Categorical semantics in Cartesian categories}

\subsection{For product types}

Given a Cartesian category $\mathcal{C}$ with $\llbracket \beta \rrbracket \in
\mathbb{C}$.

Interpret contexts $\Gamma = (x_1 : \tau_1,\ \ldots,\ x_n :
\tau_n)$ as $\llbracket \Gamma \rrbracket = \llbracket \tau_1 \rrbracket \times
\ldots \times \llbracket \tau_n \rrbracket$,

the unit type $()$ as $\llbracket () \rrbracket = 1$,

and product types $\tau_1 \ast \tau_2$ as $\llbracket \tau_1 \ast \tau_2
\rrbracket = \llbracket \tau_1 \rrbracket \times \llbracket \tau_2 \rrbracket$.

Now we ask the question: What equational theory need be imposed on the type
theory so that the interpretation is sound and complete?

In $\mathbb{C}$ the typing inference
\begin{tikzcd}
    \llbracket \Gamma \rrbracket \arrow[rr, "\llbracket \Gamma \vdash t : \tau
    \rrbracket"] \arrow[rr, "\text{(projection)}"'] &  & \llbracket \tau
    \rrbracket
\end{tikzcd}
is given by projection.

Specifically, we define:
\begin{align*}
    \llbracket \Gamma \vdash x_i : \tau_i \rrbracket &\eqdef
    \pi_i : \llbracket \tau_1 \rrbracket \times \ldots \times \llbracket \tau_n
    \rrbracket \rightarrow \llbracket \tau_i \rrbracket
    \\
    \llbracket \Gamma \vdash (t_1, t_2) : \tau_1 \ast \tau_2 \rrbracket &\eqdef
    \big(\,\llbracket \Gamma \vdash t_1 : \tau_1 \rrbracket,\,\,
           \llbracket \Gamma \vdash t_2 : \tau_2 \rrbracket\,\big)
    \\
    \llbracket \Gamma \vdash \text{fst}(t) : \tau_1 \rrbracket &\eqdef
    \pi_1 \circ \llbracket \Gamma \vdash t : \tau_1 \ast \tau_2 \rrbracket
    \\
    \llbracket \Gamma \vdash \text{snd}(t) : \tau_2 \rrbracket &\eqdef
    \pi_2 \circ \llbracket \Gamma \vdash t : \tau_1 \ast \tau_2 \rrbracket
\end{align*}

\subsection{What does soundness and completeness mean?}

Soundness means that
\begin{align*}
    \text{if}\quad& \Gamma \vdash t = u : \tau \implies \llbracket \Gamma \vdash t
    : \tau \rrbracket
    \\
    \text{and}\quad& \Gamma \vdash t = u : \tau \implies \llbracket \Gamma \vdash u
    : \tau \rrbracket
    \\
    \text{then}\quad& \llbracket \Gamma \vdash t : \tau \rrbracket = \llbracket
    \Gamma \vdash u : \tau \rrbracket
\end{align*}
and completeness means
\begin{align*}
    &\forall \Gamma \vdash t : \tau, \forall \Gamma \vdash u : \tau, \forall
    \text{Cartesian closed categories } \mathbb{C},
    \\
    &\mathbb{C}\llbracket \Gamma \vdash t : \tau \rrbracket =
    \mathbb{C}\llbracket \Gamma \vdash u : \tau \rrbracket \implies
    \Gamma \vdash t = u : \tau
\end{align*}

\subsection{Equational theroy for product types}

So, what equational theory do we need to impose on product types so that the
interpretation is sound and complete?
\begin{align*}
    \text{fst}((t_1, t_2)) =\; &t_1 : \tau_1
    \\
    \text{snd}((t_1, t_2)) =\; &t_2 : \tau_2
    \\
    t = (\text{fst}(t), \text{snd}(t)) :\; &\tau_1 \ast \tau_2
\end{align*}

Note that it is not always possible to find a (finite) set of rules.
