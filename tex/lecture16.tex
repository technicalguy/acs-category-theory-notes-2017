\chapter{Categorical Semantics}
\lecturedetails{28 November 2017}{M Fiore, A Hammond, S Steenkamp}

Type theories extend algebraic theories, and the relation between them can be
made mathematically precise. We can extend even further to get dependent type
theory.
\begin{align*}
    \text{\textbf{Algebraic theories}}\quad &\hookrightarrow\quad
    \text{\textbf{Type theories}} \\
    \text{sorts}\quad &\hookrightarrow\quad \text{types} \\
    \text{operators}\quad &\leadsto\quad \text{operators} \\
    \text{terms}\quad &\mathrel{\phantom{\hookrightarrow}}\quad \text{variable
    binding} \\
    \text{equational logic}\quad &
\end{align*}
Sorts have no structure, just names, whereas types have structure.

\section{Sorts (a.k.a. basic types)}

\newcommand{\nat}{\textbf{nat}}
\newcommand{\bool}{\textbf{bool}}
\newcommand{\ctxt}{\text{Ctxt}}

Consider a simple language with some basic types $\beta$ built in (e.g. \nat,
\bool), which may also have some operators (e.g. $+$, $\&\&$, etc.). Then we can
define contexts which give types to variables
\begin{align*}
    \Gamma = (x_1 : \beta_1,\ \ldots,\ x_n : \beta_n)
\end{align*}
The context can be thought of as a list of variables with associated types. If
each variable is numbered (var1, var2, \ldots), then the context can just be a
list of types $(\beta_1,\ \beta_2,\ \ldots)$, with the position in the list
indicating which variable the type is associated to.

The rules for the context are:
\begin{align*}
    \infer[\text{empty context}]{\bullet : \ctxt}{} &&
    \infer[\text{extra ``assumption''}]
    {(\Gamma,\ x : \beta) : \ctxt}{\Gamma : \ctxt}
\end{align*}
The only terms that are valid are those variables that are in the context:
\begin{align*}
    x_1 : \beta_1,\ \ldots,\ x_n : \beta_n \quad \vdash \quad x_i : \beta_i
    && \text{(for $i = 1, \ldots, n$)}
\end{align*}
since the only rule for typing is
\begin{align*}
    \infer[(i \in (1, \ldots, n))]{x_1 : \beta_1,\ \ldots,\ x_n :
    \beta_n\ \vdash\ x_i : \beta_i}{}
\end{align*}
