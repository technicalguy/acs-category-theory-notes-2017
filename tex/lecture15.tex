\chapter{Categorical Semantics of Dependent Sums and Products; Monads}
\lecturedetails{23 November 2017}{M Fiore, N Alcock}

\section{Categorical Semantics of Dependent Sums and Products}
There is a unique map $X \to 1$, so $\slice{\cat{C}}{1}: (X, X \to 1)$ is equivalent to simply giving $X$. Therefore $\slice{\cat{C}}{1} \iso \cat{C}$.

\begin{center}
\begin{tikzcd}
\slice{\cat{C}}{A} \arrow[rr, "(A \Rightarrow -) = \prod_A"{name=U}, bend left=49] \arrow[rr, "\sum_A"'{name=D}, bend right=49] &  & \cat{C} \arrow[ll, "-\times A"{name=M}] &  & A \times C \arrow[d] \arrow[r] \arrow[dr, phantom, "\lrcorner", very near start] & C \arrow[d] \\
 &  &  &  & A \arrow[r] & 1
\arrow[from=U, to=M, phantom, "\dashv" rotate=90]
\arrow[from=M, to=D, phantom, "\dashv" rotate=90]
\end{tikzcd}
\end{center}

\begin{example}
In $\catset$ we have: $I \xto{f} J$.

\begin{center}
\begin{tikzcd}
\slice{\catset}{I} \arrow[rr, "\prod_f"{name=U}, bend left=49] \arrow[rr, "\sum_f"'{name=D}, bend right=49] &  & \slice{\catset}{J} \arrow[ll, "f^{*}"{name=M}]
\arrow[from=U, to=M, phantom, "\dashv" rotate=90]
\arrow[from=M, to=D, phantom, "\dashv" rotate=90]
\end{tikzcd}
\end{center}

\begin{align*}
\{X_i\}_{i \in I} \xmapsto{\sum_f} \{\sum_{i : f(i) \neq j} X_i\}_{j \in J} \\
\{Y_{f(i)}\}_{i \in I} \overset{f^{*}}{\mapsfrom} \{Y_j\}_{j \in J} \\
\{X_i\}_{i \in I} \xmapsto{\prod_f} \{\prod_{i : f(i) \neq j} X_i\}_{j \in J}
\end{align*}

$f^{*}$ is sometimes known as \textit{reindexing}.
\end{example}

Consider  $\slice{\catset}{I}$. Objects are sets $X$ with maps into $I$: $(X, X \xto{p} I)$. $X = \bigcup_{i \in I} \{x \in X \suchthat p(x) = i\}$. To make this precise, we can consider:

\begin{center}
\begin{tikzcd}
\slice{\catset}{I} \arrow[rr, bend left=49] &  & \catset^I \arrow[ll, bend left=49]
\end{tikzcd}
\end{center}

where $\catset^I$ is a category with objects $X = \{X_i\}_{i \in I}$ and morphisms such that for every morphism $S \xto{\varphi} T$ in $\catset$, we have a morphism $S_i \xto{\varphi_i} T_i$ for each $i \in I$.

\begin{center}
\begin{tikzcd}
X \arrow[d, "p"] \arrow[r, maps to] & \{p^{-1}(i)\}_{i \in I} = X_i &  & \bigcup_{i \in I} (\{i\} \times S_i) = \biguplus_{i \in I} S_i \arrow[d] &  & S=\{S_i\}_{i \in I} \arrow[ll, maps to] \\
I &  &  & I &  & 
\end{tikzcd}
\end{center}

\section{Equivalences of categories}

\begin{definition}[Equivalence of categories]
Let $F: \cat{C} \to \cat{D}$ and $G: \cat{D} \to \cat{C}$ be adjoint functors, $F \adjoint G$. If there exist natural transformations such that $\Id{\cat{C}} \iso GF$ and $FG \iso \Id{\cat{D}}$, then $\cat{C}$ and $\cat{D}$ are said to be equivalent.
\end{definition}

$\slice{\catset}{I}$ and $\catset^{I}$ are equivalent.

\begin{center}
\begin{tikzcd}
X \arrow[r, maps to] \arrow[d] & \{p^{-1}(i)\}_{i \in I} \arrow[r, maps to] & \biguplus_{i \in I} p^{-1}(i) \arrow[r, "\cong" description, phantom, no head] \arrow[d] & X \arrow[ld] \\
I &  & I &  \\
\{S_i\}_{i \in I} \arrow[r, maps to] & \biguplus_{i \in I} S_i \arrow[r, maps to] \arrow[d] & \{\{i\} \times S_i\}_{i \in I} \arrow[r, "\cong" description, phantom, no head] & \{S_i\}_{i \in I} \\
 & I &  & 
\end{tikzcd}
\end{center}

Equivalence is the natural notion for comparing categories.

For $\scat{C}, \scat{D}$ small categories:

\begin{center}
\begin{tikzcd}
\catset^{\scat{C}^{op}} \arrow[rr, "f_{*}"{name=U}, bend left=49] \arrow[rr, "f!"'{name=D}, bend right=49] &  & \catset^{\scat{D}^{op}} \arrow[ll, "f^{*}"{name=M}] \\
\scat{C} \arrow[rr, "f"'] \arrow[u, hook, "y"] & & \scat{D} \arrow[u, hook, "y"]
\arrow[from=U, to=M, phantom, "\dashv" rotate=90]
\arrow[from=M, to=D, phantom, "\dashv" rotate=90]
\end{tikzcd}
\end{center}

where $f^{*}$ maps $(\scat{D}^{op} \xto{X} \catset) \mapsto (\scat{C} \xto{Xf} \catset)$, and $y$ is the Yoneda embedding. This is the categorical version of:

\begin{center}
\begin{tikzcd}
\powerset(A) \arrow[rr, ""{name=U}, bend left=49] \arrow[rr, ""{name=D}, bend right=49] &  & \powerset(B) \arrow[ll, ""{name=M}] \\
A \arrow[rr, "f"'] \arrow[u, hook, "\{-\}"] & & B \arrow[u, hook, "\{-\}"']
\arrow[from=U, to=M, phantom, "\dashv" rotate=90]
\arrow[from=M, to=D, phantom, "\dashv" rotate=90]
\end{tikzcd}
\end{center}

\section{Adjoint functors}
To recap, the following three conditions are equivalent:
\begin{align*}
\cat{D}(FX, Y) \natiso \cat{C}(X, GY) \\
X \xto{\eta_X} GFX & \text{(univ.)} \\
FGY \xto{\eta_Y} Y & \text{(univ.)} \\
\end{align*}

For $X \to Y$, if $\eta$ and $\varepsilon$ are natural transformations, then the following diagram commutes:

\begin{center}
\begin{tikzcd}
X \arrow[r, "\eta_X"] \arrow[d, "f"'] & GFX \arrow[d, "GFf"] \\
Y \arrow[r, "\eta_Y"] & GFY
\end{tikzcd}
\end{center}

There is a general proof technique for proving diagrams commute using adjunctions. Say that we want to prove the following diagram commutes:

\begin{center}
\begin{tikzcd}
X \arrow[rd, bend left=49] \arrow[rd, bend right=49] &  \\
 & GZ
\end{tikzcd}
\end{center}

This diagram commutes iff the following one does:

\begin{center}
\begin{tikzcd}
X \arrow[rd, "F", bend left=49] \arrow[rd, "F"', bend right=49] &  &  \\
 & FGZ \arrow[rd, "\varepsilon_Z"] &  \\
 &  & Z
\end{tikzcd}
\end{center}

Similarly, if we want to prove the following diagram commutes:

\begin{center}
\begin{tikzcd}
FA \arrow[rd, bend left=49] \arrow[rd, bend right=49] &  \\
 & B
\end{tikzcd}
\end{center}

we can use the commutativity of the following one:

\begin{center}
\begin{tikzcd}
A \arrow[rd, "\eta_A"] &  &  \\
 & GFA \arrow[rd, "G", bend left=49] \arrow[rd, "G"', bend right=49] &  \\
 &  & GB
\end{tikzcd}
\end{center}

\begin{example}
$ $ \newline
\begin{center}
\begin{tikzcd}
FX \arrow[r, "F\eta_X"] \arrow[d, "Ff"] & FGFX \arrow[d, "FGFf"] &  \\
FY \arrow[r, "F\eta_Y"] & FGFY \arrow[rd, "\varepsilon_Y"] &  \\
 &  & FY
\end{tikzcd}
\end{center}

\[\Efrac{FX' \xto{Fh} FX \xto{f} Y}{X' \xto{h} X \xto{\hat{f}} G}\]
\[(f \comp Fh)\hat{} = \hat{f} \comp h\]
So:
\[\Efrac{FX \xto{\id{FX}} FX}{X \xto{\eta_X \iso \hat{\id{FX}}} GFX}\]

\begin{definition}[Adjunction triangle laws]
For $F \adjoint G$, the following diagrams commute (making use of the proof technique above):

\begin{center}
\begin{tikzcd}
G \arrow[Rightarrow, rd, "\id{G}"'] \arrow[Rightarrow, r, "\eta G"] & GFG \arrow[Rightarrow, d, "G\varepsilon"] & F \arrow[Rightarrow, rd, "\id{F}"'] \arrow[Rightarrow, r, "F\eta"] & FGF \arrow[Rightarrow, d, "\varepsilon F"] \\
 & G &  & F
\end{tikzcd}
\end{center}

In fact, to have an adjunction is equivalent to having:
\begin{center}
\begin{tikzcd}
\cat{C} \arrow[r, "F", bend left] & \cat{D} \arrow[l, "G", bend left]
\end{tikzcd}
\end{center}
together with $\eta: \Id{} \natto GF$ and $\varepsilon: FG \natto \Id{}$ such that the triangle laws hold.
\end{definition}
\end{example}

\section{Monads}
We want to compare $(FX \xto{f} Y) \mapsto (X \xto{\eta_X} GFX \xto{Gf} GY)$ and $(FX \xto{Fg} FGY \xto{\varepsilon_Y} Y) \mapsfrom (X \xto{g} GY)$ to the identity.

\begin{definition}[Monad]
A monad on $\cat{C}$ is an endofunctor $T$ together with $\eta: \Id{} \natto T$, $\mu: TT \natto T$ satisfying the monoid laws:

\textbf{Identity:}
\begin{center}
\begin{tikzcd}
T \arrow[rd, "\id{}"] \arrow[r, "T\eta"] & TT \arrow[d, "\mu"] & T \arrow[ld, "\id{}"'] \arrow[l, "\eta T"'] \\
 & T & 
\end{tikzcd}
\end{center}

\textbf{Associativity:}
\begin{center}
\begin{tikzcd}
TTT \arrow[r, "T\mu"] \arrow[d, "\mu T"'] & TT \arrow[d, "\mu"] \\
TT \arrow[r, "\mu"] & T
\end{tikzcd}
\end{center}
\end{definition}

If we pick $T = GF$, then the monoid laws are satisfied by the adjunction's triangle laws (for example, $TT \xto{\mu} T$ is equal to $GFGF \xto{G \varepsilon F} GF$).

\begin{proposition}
Every adjunction, $F: \cat{D} \to \cat{C} \adjoint G: \cat{C} \to \cat{D}$ induces a monad on $\cat{C}$, where $T = GF$.

\begin{center}
\begin{tikzcd}
\cat{D} \arrow[r, "F"{name=F}, bend left] & \cat{C} \arrow[l, "G"{name=G}, bend left] \arrow[loop right, "T"]
\arrow[from=F, to=G, phantom, "" description]{}{\perp}
\end{tikzcd}
\end{center}
\end{proposition}