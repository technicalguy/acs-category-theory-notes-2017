\chapter{Natural transformations}
\lecturedetails{31 October 2017}{M Fiore, V Chakraborty, N Licker}

Natural transformations are morphisms between functors with the same domain
and codomain.  

\section{Motivation}

In the category $\catpreo$, morphisms are monotone functions between
preorders.  These morphisms can in turn be pre-ordered:
\[
  f \le g : (P, \le_P) \to (Q, \le_Q)
  \iffdef \forall p \in P. f(p) \le g(p) \enspace.
\]

For categories, with morphisms between them given by functors, we have natural
transformations 
\[
  \varphi: F \Rightarrow G: \lscat A\to \lscat B
\]
that relate them.

\begin{center}
    \begin{tikzcd}[sep=6em]
      \lscat A
        \ar[bend left=40, "F"{name=T, above}]{r}
        \ar[bend right=40, "G"{name=B, below}]{r}
      &
      \lscat B
      \\
      \ar[Rightarrow, from=T, to=B, "{\varphi}", shorten <= 2pt, shorten >= 2pt]
    \end{tikzcd}\\[3mm]
\end{center}

\section{Definition}

\begin{definition}[Natural Transformations]

A natural transformation $\varphi: F \Rightarrow G : \lscat A \to \lscat B$ is
a family:
\[
  \varphi = \Big \{ \varphi_A : F A \to G A \Big \}_{A \in \Ob(\lscat A)}
\]
such that the following \emph{naturality} condition holds:
\begin{center}
    for all
    \enspace
    \begin{tikzcd}
        A \arrow [d, "f"] \\ A' 
    \end{tikzcd} 
    in $\lscat A$,
    the square 
    \enspace
      \begin{tikzcd}
        F A \arrow[r, "\varphi_A"] \arrow [d, "F f",swap] &
        G A \arrow[d, "G f"]
        \\
        F A' \arrow[r, "\varphi_{A'}",swap] &
        G A'
    \end{tikzcd}
    commutes in $\lscat B$
    \enspace.
\end{center}

\end{definition}

\section{Examples}

\subsection{For the $\fnclist$ functor}

Recall that the list functor \(\fnclist: \catset \to \catset\) is defined by:

\begin{center}
\begin{tikzcd}
A
\arrow[dd, "f"{name=L, right}] & & List(A)
\arrow[dd, "\fnclist(f)"{name=R, left}]
\\
\arrow[rr, "\fnclist{}", from=L, to=R, start anchor={[yshift=-0.2ex]}, mapsto]
&  &
\\
B &  & \fnclist(B)
\end{tikzcd}
\end{center}

where $\fnclist(f)$ is the unique homomorphism
$$
(List(A), nil, @) \to (List(B), nil, @)
$$
such that the following square commutes:
\begin{center}
\begin{tikzcd}
A
\arrow[d, "f",swap]
\arrow[r, "{[\phold]_A}"]
&
List(A)
\arrow[d, "List(f)"]
\\
B
\arrow[r, "{[\phold]_B}", swap]
&
List(B)
\end{tikzcd}
\end{center}

\begin{example}[List-singletons]
The family of list-singleton functions
$$
{[\phold]} 
= 
\bigsetof{\, {[\phold]}_A : A \to List(A) \,}_{A\in\catset}
$$

is a natural transformation from $\fncid{\catset}$ to $\fnclist$:

\begin{center}
    \begin{tikzcd}[sep=6em]
      \catset
        \ar[bend left=40, "\fncid{\catset}"{name=T, above}]{r}
        \ar[bend right=40, "\fnclist"{name=B, below}]{r}
      &
      \catset
      \\
      \ar[Rightarrow, from=T, to=B, "{[\phold]}", shorten <= 2pt, shorten >= 2pt]
    \end{tikzcd}\\[3mm]
\end{center}
\end{example}


\begin{example}[List flattening]
Let $\text{flat}_A : \fnclist\big(\fnclist(A)\big) \to \fnclist(A)$ be the
function that flattens lists.

The family of functions
\[
  \text{flat} 
  = \bigsetof{\, 
      \text{flat}_A: \fnclist\big(\fnclist(A)\big) \to \fnclist(A) 
    \,}_{A \in \catset}
\]
is a natural transformation:
\begin{center}
    \begin{tikzcd}[sep=6em]
      \catset
        \ar[bend left=40, "List \circ List"{name=T, above}]{r}
        \ar[bend right=40, "\fnclist"{name=B, below}]{r}
      &
      \catset
      \\
      \ar[Rightarrow, from=T, to=B, "{[\phold]}", shorten <= 2pt, shorten >= 2pt]
    \end{tikzcd}\\[3mm]
\end{center}
\end{example}

\begin{exercise}
Show that $\text{flat}$ is a natural transformation using its universal
definition as follows:
\begin{center}
    \begin{tikzcd}[sep=6em]
      \fnclist(A)
        \ar["{[\phold]}_{\fnclist(A)}"]{r}
        \ar["id_{\fnclist(A)}"below]{rd}
      &
      \fnclist\big(\fnclist(A)\big)
        \ar[" (id_{\fnclist(A)})^\# \ \eqdef \ \text{flat}_A"]{d}
      \\
      &
      List(A)
    \end{tikzcd}\\[3mm]
\end{center}
\end{exercise}

\subsection{Coproducts}

Let $\lscat C$ be a category with binary coproducts (or sums); that is, we
have 
\begin{center}
    \begin{tikzcd}%[sep=6em]
      &
      A + B
        \ar[dotted, "{\exists![f,g]}"]{dd}
      &
      \\
      A
        \ar["f",swap]{dr}
        \ar["\iota_1^{A,B}"above]{ru}
      &
      &
      B
        \ar["g"]{ld}
        \ar["\iota_2^{A,B}"above]{lu}
      \\
      & \overset{\mbox{$C$}}{\forall} & 
    \end{tikzcd}\\[3mm]
\end{center}
for every pair of objects $A,B$ in $\lscat C$.

We have a \emph{coproduct functor} $\lscat C \times \lscat C \to \lscat C$
mapping as follows 
\begin{center}
  \begin{tikzcd}
    (A,B) \ar["{(f,g)}"{right},name=L]{d} 
    & 
    A+B \ar["f+g"{left},name=R]{d} 
    \ar[mapsto,from=L,to=R,shorten >= 0pt,shorten <=30pt]
    \\
    (A',B') & 
    A'+B'
  \end{tikzcd}
\end{center}
and defined by:
\begin{center}
    \begin{tikzcd}[sep=3em]
      &
      A + B
        \ar[dotted, "{\exists!(f+g)}"]{ddd}
      &
      \\
      A
        \ar["f"]{d}
        \ar["\iota_1^{A,B}"above]{ru}
      &
      &
      B
        \ar["g"]{d}
        \ar["\iota_2^{A,B}"above]{lu}
      \\
      A'
        \ar["\iota_1^{A',B'}"below]{rd}
      &
      &
      B'
        \ar["\iota_2^{A',B'}"below]{ld}
      \\
      &
      A' + B'
      &
    \end{tikzcd}\\[3mm]
\end{center}

\begin{exercise}
Show that the families 
\[
\iota_i
=
\bigsetof{\,
  \iota_i^{A_1,A_2}: A_i \to A_1 + A_2 
  \,}_{(A_1,A_2) \in \lscat C \times \lscat C}
\qquad(i=1,2)
\]
are natural transformations:
\begin{center}
    \begin{tikzcd}[sep=4em]
      \lscat C \times \lscat C
        \ar[bend left=40, "\pi_i"{name=T, above}]{r}
        \ar[bend right=40, "+"{name=B, below}]{r}
      &
      \lscat C
      \\
      \ar[Rightarrow, from=T, to=B, "{\iota_i}", shorten <= 2pt, shorten >= 2pt]
    \end{tikzcd}
\end{center}
\end{exercise}


\begin{definition}[Codiagonal]
For $A \in \Ob(\lscat C)$, we define the \emph{codiagonal} 
$\nabla_A:A + A\to A$ as the unique map such that
\begin{center}
    \begin{tikzcd}
      &
      A + A
        \ar[dotted, "{\exists!\nabla_A}"]{dd}
      &
      \\
      A
        \ar["id"below]{rd}
        \ar["\iota_1^{A,A}"above]{ru}
      &
      &
      A
        \ar["id"]{ld}
        \ar["\iota_2^{A,A}"above]{lu}
      \\
      &
      A
      &
    \end{tikzcd}\\[3mm]
\end{center}
\end{definition}

\begin{exercise}
Show that 
$\nabla = \Big\{ \nabla_A:A + A \to A \Big\}_{A \in \lscat C}$ is a natural
transformation:  
\begin{center}
    \begin{tikzcd}
      &
      \lscat C \times \lscat C
        \ar["+"above]{rd}
      &
      \\
      \lscat C
        \ar[bend right=30, "\Id{\lscat C}"{name=B, below}]{rr}
        \ar["\Delta"above]{ru}
      &
      &
      \lscat C
      \arrow[Rightarrow, from=1-2,to=B,"\nabla", shorten <= 2pt, shorten >= 2pt]
    \end{tikzcd}\\[3mm]
\end{center}
where $\Delta:\lscat C\to\lscat C\times\lscat C$ is the 
\emph{diagonal functor} given by $\Delta(-)\eqdef(-,-)$.
\end{exercise}

\section{Functor Categories}

\begin{definition}[Functor Category]
Given categories $\lscat X$ and $\lscat A$, the \emph{functor category} 
$\lscat A^\lscat X$ \big(or $[\lscat X, \lscat A]$\big) is the category whose
objects are functors from $\lscat X$ to $\lscat A$ and whose morphisms in
$\lscat A^\lscat X(F, G)$ are natural transformations from $F$ to $G$.  The
identity morphism $id_F : F \Rightarrow F : \lscat X \to \lscat A$ is
$id_F = \lbrace id_{FX}: FX \to FX \rbrace_{X \in \Ob(\lscat X)}$.
Composition is defined as follows: in the situation
\[
    \begin{tikzcd}[sep=5em]
      \lscat X
        \ar[bend left=60, "F"{name=F, above}]{r}
        \ar["G"name=G]{r}
        \ar[bend right=60, "H"{name=H, below}]{r}
      &
      \lscat A \\
      \ar[Rightarrow, from=F, to=G, "\varphi", shorten <= 2pt, shorten >= -1pt]
      \ar[Rightarrow, from=G, to=H, "\gamma", shorten <= 2pt, shorten >= 2pt]
    \end{tikzcd}\\[-12mm]
\]
we let $\gamma \circ \varphi:F\Rightarrow H$ be given by
\[
\gamma \circ \varphi 
\eqdef
\bigsetof{\, 
  \gamma_X \circ \varphi_X 
  : F X \to H X 
  \,}_{X \in \Ob(\lscat X)}
\]
that is,\\[-8mm]
\begin{center}
    \begin{tikzcd}
      F X
        \ar["(\gamma \circ \varphi)_X \,\eqdef\, \gamma_X \circ \varphi_X"]{rr}
        \ar["\varphi_X",swap]{rd}
      &
      &
      H X
      \\
      &
      G X
        \ar["\gamma_X",swap]{ru}
      &
    \end{tikzcd}
\end{center}
\end{definition}

\begin{exercise}
Check that if $\varphi$ and $\gamma$ as above are natural transformations then
so is $\gamma \circ \varphi$ as defined.
\end{exercise}

\begin{example}
Consider the terminal category $\nelem 1$ and an arbitrary locally small
category $\lscat C$. The functor category $\lscat C^{\nelem1}$ has:
\begin{itemize}
  \item 
    Functors 
    \[
      \framebox{\begin{tikzcd}
          * \arrow[out=150,in=210, loop, looseness=5, "id_*"left]
      \end{tikzcd}}
      \longrightarrow 
      \lscat C
    \]
    from $\nelem 1$ to $\lscat C$ as objects; so that 
    \[
      \Ob(\lscat C^{\nelem1})\iso\Ob(\lscat C)
      \enspace.
    \]

  \item 
    Natural transformations 
    \[
      \varphi: F \Rightarrow G: \nelem 1 \to \lscat C
    \]
    between functors as morphisms; so that
    \[
      \varphi = \bigsetof{\, \varphi_* : F * \to G * \text{ in } \lscat C \,}
      \enspace.
    \]
\end{itemize}
\end{example}

\begin{exercise}
Prove that $\lscat C^{\nelem1} \iso \lscat C$.
\end{exercise}

\begin{example}
Let $\cat D$ be the category with two objects and a parallel pair of arrows:
\[
  \begin{array}{|c|}\hline 
  \begin{tikzcd}
    E
      %\arrow[out=150,in=210, loop, looseness=5, "id_*"left]
      \arrow[shift left=+2, "s"above]{r}
      \arrow[shift left=-2, "t"below]{r}
    &
    N %\arrow[out=30,in=-30, loop, looseness=5, "id_\bullet"right]
  \end{tikzcd}
  \\ \hline \end{array}
\]
where, as it is customary, identities have been omitted from the picture.

In the functor category $\catset^{\cat D}$, 
\begin{itemize}
\item
the objects are functors
\[
  \begin{array}{|c|}\hline 
  \begin{tikzcd}
  E \arrow[shift left=+2, "s"above]{r} \arrow[shift left=-2, "t"below]{r} 
  & N
  \end{tikzcd}
  \\ \hline \end{array}
  \overset G\longrightarrow 
  \catset 
\]
and therefore essentially given by sets and functions between them as follows:
\[
  \begin{tikzcd}[sep=4em]
  G(E) \arrow[shift left=+2, "G(s)"above]{r} 
  \arrow[shift left=-2, "G(t)"below]{r} & G(N)
  \end{tikzcd}
\]

\item
the morphisms are natural transformations
\[
  \varphi : G_1 \Rightarrow G_2 : \cat D\to \catset
\]
and therefore given by functions 
\begin{align*}
  \varphi_E &: G_1(E) \longrightarrow G_2(E)
  \\[1mm]
  \varphi_N &: G_1(N) \longrightarrow G_2(N)
\end{align*}
such that the diagram
\[\begin{tikzcd}[sep=4em]
    G_1(E)
      \arrow["\varphi_E"]{r}
      \arrow[shift left=+2, "G_1(t)"right]{d}
      \arrow[shift left=-2, "G_1(s)"left]{d}
    &
    G_2(E)
    \arrow[shift left=+2, "G_2(t)"right]{d} 
    \arrow[shift left=-2, "G_2(s)"left]{d}
    \\
    G_1(N)
    \arrow["\varphi_N",swap]{r}
    &
    G_2(N)
\end{tikzcd}\]
commutes serially.
\end{itemize}
\end{example}

\begin{exercise}
Prove that $\catset^{\cat D} \iso \catdirgraph$.
\end{exercise}

