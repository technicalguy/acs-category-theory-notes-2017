\chapter{Limits and colimits}
\lecturedetails{9 November 2017}{M Fiore, S Lau, D Szamozvancev}

We can generalize products and coproducts, amongst other notions in category
theory, using the idea of limits and colimits. We begin with a motivating
example.

\section{Example: lub of $\omega$-chains}

Consider a preorder $(P, \leq)$ interpreted as a category  $\lscat{P}$.

In $\lscat{P}$,  products are meets ($\wedge$) and coproducts are joins
($\vee$). These constructions were previously defined for \emph{pairs} of
elements.

In domain theory, we are interested in other kinds of least upper bounds. In
particular, we are interested in least upper bounds of $\omega$-chains
(countably increasing chains)
\[
	x_0 \leq x_1 \leq x_2 \leq \cdots \leq x_n \leq \cdots \quad (n \in \nats)
\]

The least upper bound of a chain $x_n$, if it exists, is denoted $\lub_{n \in
    \nats}$ and in $\lscat{P}$ is analogous to the ``least upper bound of the
join construction''.

In order to generalize these kinds of constructions in a category, we need
something analogous to the $\omega$-chain and to taking the least upper bound.

\subsection{Colimits of $\omega$-chains}

\begin{definition}[$\omega$-chain]
Informally, an $\omega$-chain in a category is given by a family of objects
$X_i$ with a map from $X_i$ to $X_{i+1}$ for $i \in \nats$.
\[
    X_0 \morpharrow X_1 \morpharrow X_2 \morpharrow \cdots \morpharrow X_i
    \morpharrow \cdots
\]

Formally, consider the category $\underline{\omega}$ with objects the natural
numbers and arrows $m \morpharrow n$ iff $m \leq n$. An $\omega$-chain in a
category $\lscat{C}$ is nothing but a functor $X : \underline{\omega} \to
\lscat{C}$ such that
\begin{align*}
	n & \mapsto X_n & (\text{object map}) \\
	(m \leq n) & \mapsto (X_m \morpharrow X_n) &(\text{morphism map})
\end{align*}

\end{definition}

\begin{definition}[Colimits of $\omega$-chains]
The colimit of an $\omega$-chain
\[
    X_0 \morpharrow X_1 \morpharrow X_2 \morpharrow \cdots \morpharrow X_i
    \morpharrow \cdots
\]
consists of an (initial universal) object $C$ (called a \emph{cocone}, with dual
\emph{cone} for limits) and a family of arrows $\langle \gamma_n : X_n
\morpharrow C \rangle _{n \in \nats}$

\begin{center}
\begin{tikzcd}[row sep=2cm, column sep=2cm]
X_0    \arrow[r]
	   \arrow[drr,"\gamma_0",swap] &
X_1    \arrow[r]
	   \arrow[dr,"\gamma_1",swap]  &
\cdots \arrow[r]
	   \arrow[d,dashed]            &
X_n    \arrow[r]
	   \arrow[dl, "\gamma_n"]      &
\cdots \arrow[dll, dashed]
\\
&& C
\end{tikzcd}
\end{center}

such that the following diagram commutes $\forall n \in \nats$.

\begin{center}
\begin{tikzcd}[column sep=small]
X_n     \arrow[rr]
	    \arrow[dr, "\gamma_n", swap] &&
X_{n+1} \arrow[dl, "\gamma_{n+1}"]
\\
& C
\end{tikzcd}
\end{center}

The cocone $(C, \gamma_n)$ is initial universal, which means that for any other
object $D$ and family of arrows
$\langle \delta_n : X_n \rightarrow D\rangle_{n \in \nats}$ such that the
following diagram commutes,
\begin{center}
\begin{tikzcd}[column sep=small]
X_n     \arrow[rr]
	    \arrow[dr, "\delta_n", swap] &&
X_{n+1}
	    \arrow[dl, "\delta_{n+1}"]
\\
& D
\end{tikzcd}
\end{center}

there exists a unique map
\[
	C \morpharr{\exists ! u} D
\]
such that the following commutes for all $n \in \nats$.
\begin{center}
\begin{tikzcd}[column sep=small]
& X_n \arrow[dl, "\gamma_n",swap]
	  \arrow[dr, "\delta_n"]
\\
C     \arrow [rr, "u",swap]
&& D
\end{tikzcd}
\end{center}

Intuitively, this universal property captures the notion of ``least'' upper
bound in the context of preorders. For if $D$ is also an upper bound, then there
is a (unique) arrow from $C$ to $D$, expressing ``$C \leq D$''.

\end{definition}

This idea of a colimit generalizes to other diagrams. For instance, consider the
following expressing countable sums (the colimit is denoted by $\coprod Y_k$).
\begin{center}
\begin{tikzcd}[column sep=small]
& {Y_n\ (n \in \nats)}
	\arrow[dl, "\iota_k",swap]
	\arrow[dr, "\delta_n"]
\\
\coprod_{k \in \nats} Y_k
	\arrow [rr, "\exists ! u",swap, dashed]
&& D
\end{tikzcd}
\end{center}

N.B.~$Y_n$ is discrete, whereas the $\omega$-chain used above in defining
colimits had structure. We can define limits and colimits for arbitrary
diagrams, though they may not necessarily exist.
