\chapter{Functors (definition)}
\lecturedetails{24 October 2017}{M Fiore, J Clarke, T Parks}

\section{Duality}
\begin{definition}[Duality]
Given a category $\mathcal{C}$, we define the \emph{opposite} or \emph{dual}
category, $\mathcal{C}^{\op}$, to be the category with:
    \begin{itemize}
        \item $\Ob(\mathcal{C}^{\op}) = \Ob(\mathcal{C})$
        \item $\mathcal{C}^{\op}(A, B) \eqdef \mathcal{C}(B, A)$
    \end{itemize}

\begin{minipage}[t]{0.49\textwidth}
    \begin{center}
        In $\mathcal{C}^{\op}$

        \vspace{15pt}

        \begin{tikzcd}
            A \arrow[d, swap, "f"] \\
            B
        \end{tikzcd}
    \end{center}
\end{minipage}%
\vline%
\begin{minipage}[t]{0.49\textwidth}
    \begin{center}
        In $\mathcal{C}$

        \vspace{15pt}

        \begin{tikzcd}
            A \\
            B \arrow[u, "f"]
        \end{tikzcd}
    \end{center}
\end{minipage}

Identities are the same as in $\mathcal{C}$, that is:

\begin{equation*}
    \id{A}^{\op} \eqdef \id{A}
\end{equation*}

For composition, we need:

\begin{equation*}
    \equalto{\mathcal{C}^{\op}(B, C)}{\mathcal{C}(C, B)} \times
    \equalto{\mathcal{C}^{\op}(A, B)}{\mathcal{C}(B, A)}
    \xrightarrow{\comp^{\op}_{A,B,C}}
    \equalto{\mathcal{C}^{\op}(A, C)}{\mathcal{C}(C, A)}
\end{equation*}

Therefore, we define:

\begin{equation*}
    g \comp^{\op}_{A,B,C} f = f \comp_{C,B,A} g
\end{equation*}
\end{definition}

\begin{definition}[Monomorphism]
A \emph{monomorphism} in a category $\mathcal{C}$ is a map $f : A \to B$ such
that for all $X \in \Ob(\mathcal{C})$ and for all $a, a' : X \to A$,
$f \comp a = f \comp a' \implies a = a'$.
\end{definition}

\begin{definition}[Epimorphism]
An \emph{epimorphism} in a category $\mathcal{C}$ is a monomorphism in
$\mathcal{C}^{\op}$, \ie~$f : A \to B$ is an epimorphism if:
\[
    \forall T \in \Ob(\mathcal{C}) . \forall t, t' : B \to T .
    t \comp f = t' \comp f \implies t = t'
\]
Here, $T$ can be thought of as an object of ``tests''.
\end{definition}

\begin{exercise}
Show that, in $\catset$, epimorphisms are the same as surjections.
\end{exercise}

\begin{exercise}
Show that each of isomorphisms, epimorphisms, monomorphisms, sections and
retractions are closed under composition; \ie~if $f : A \to B$ and
$g : B \to C$ are in one of these classes, then so is $g \comp f : A \to C$.
\end{exercise}

\begin{exercise}
Prove the labelled inclusions:

\begin{center}
    \begin{tikzpicture}
        \matrix[matrix of math nodes, name=m, column sep=0.6em, row sep=1.2em] {
            &
            |[alias=Sec]| \mathcal{C}_{\mathrm{sections}} &
            |[alias=Mono]| \mathcal{C}_{\mathrm{monos}} &
            \\
            |[alias=Cong]| \mathcal{C}_{\mathrm{isos}} &
            |[alias=Ex1]| &
            |[alias=Ex2]| &
            |[alias=C]| \mathcal{C}
            \\
            &
            |[alias=Ret]| \mathcal{C}_{\mathrm{retractions}} &
            |[alias=Epi]| \mathcal{C}_{\mathrm{epis}} &
            \\
        };

        \node (Ex) [fit=(Ex1)(Ex2)]{exercises};

        \path (Cong) edge[draw=none] node[sloped, allow upside down, auto=false] {$\subseteq$} (Sec)
              (Cong) edge[draw=none] node[sloped, allow upside down, auto=false] {$\subseteq$} (Ret)
              (Sec) edge[draw=none] node (SecMono) [sloped, allow upside down, auto=false] {$\subseteq$} (Mono)
              (Ret) edge[draw=none] node (RetEpi) [sloped, allow upside down, auto=false] {$\subseteq$} (Epi)
              (Mono) edge[draw=none] node[sloped, allow upside down, auto=false] {$\subseteq$} (C)
              (Epi) edge[draw=none] node[sloped, allow upside down, auto=false] {$\subseteq$} (C)
              (Ex) edge (SecMono)
              (Ex) edge (RetEpi);
    \end{tikzpicture}
\end{center}
\end{exercise}

Recall that the product of a family of objects $\setof{A_i \suchthat i \in I}$
is given by:

\begin{equation*}
    \left(\prod_{i\in I} A_i\right) \xrightarrow{\pi_k} A_k
    \qquad (k\in I)
\end{equation*}

such that it is a terminal universal solution.

\begin{definition}
\emph{Coproducts}, or \emph{sums}, in $\mathcal{C}$ are products in
$\mathcal{C}^{\op}$. The coproduct of a family
$\setof{A_i \suchthat i \in I}$ is given by

\begin{center}
    \begin{tikzcd}
        A_k \arrow[r, "\iota_k"] \arrow[dr, swap, "\forall f_k"] &
        \left(\coprod_{i \in I} A_i\right) \arrow[d, "\exists!f_k"]
        \\
        &
        X
    \end{tikzcd}
    \qquad$(k\in I)$
\end{center}
\end{definition}

In $\catset$, coproducts are disjoint unions:

\begin{equation*}
    \begin{aligned}
        \coprod_{i \in I} A_i \ = \ \bigcup_{i \in I}\, \setof{i} \times A_i &
        \overset{\iota_k}{\leftarrow} A_k
        \\
        (k, a) &
        \mapsfrom a
    \end{aligned}
\end{equation*}

\begin{exercise}
Calculate coproducts/sums in the categories introduced so far ($\catpfn$,
$\catrel$, $\catpred$, $\catpreo$, $\catmon$, \dots).
\end{exercise}

\begin{exercise}
Prove or disprove whether bijections ($\iffdef$ monomorphic and epimorphic) are
the same as isomorphisms.
\end{exercise}

\section{Functors}

Categories are mathematical structures with their own morphisms,
\emph{functors}:
\begin{center}
\begin{tikzcd}
    \mathcal{A} \arrow[r, "F"]  & \mathcal{B}
\end{tikzcd}
\end{center}
We will motivate their definition by considering two special kinds of
categories, monoids and preorders, together with their respective morphisms.

\subsection{Monoids can be regarded as categories}

A monoid $(M, \shortmid, \star)$ maybe seen as the category $\mathcal M$ with
a single object:
\begin{align*}
\Ob(\mathcal M) &\eqdef \setof{ \bullet }
\end{align*}
for which the hom-set consists of the elements of the monoid:
\begin{align*}
\mathcal{M}(\bullet, \bullet) &\eqdef M
\end{align*}
The identity is the neutral element of the monoid:
\begin{align*}
\id{\bullet} &\eqdef\ \shortmid
\end{align*}
and the composition of morphisms is the monoid multiplication:
\begin{align*}
f \comp g &\eqdef f \star g
\enspace.
\end{align*}

Conversely, every category $\mathcal C$ with a singleton set of objects, say
$\setof{\bullet}$, has an underlying monoid structure given by
\[
  \big(\, \mathcal{C}(\bullet,\bullet) \,,\, \id{\bullet} \,,\, \comp \,\big)
\]

The above two constructions are in bijective correspondence.

\subsection{Preorders can be regarded as categories}

For a preorder $(P, \leq)$, consider the category $\mathcal P$ with
$\Ob(\mathcal P) = P$ and hom-sets
\[
\mathcal{P}(p, q)
= \begin{cases}
    \bigsetof{\, (p,q) \,} & \text{if } p \leq q
    \\[1mm]
    \emptyset & \text{otherwise }
  \end{cases}
\qquad
(p,q \in P)
\]

As $p \leq p$, the identity is given by
\begin{align*}
\id{p} &\eqdef (p,p)
\end{align*}

We also need to consider the operation of composition:
\[
\mathcal{P}(q, r) \times \mathcal{P}(p, q) \to \mathcal{P}(p, r)
\]
We can determine it by cases:
\begin{itemize}
\item
If either $\mathcal{P}(q, r)$ or $\mathcal{P}(p, q)$ are empty, then so is
$\mathcal{P}(q, r) \times \mathcal{P}(p, q)$ and the composition map is
uniquely determined.

\item
Otherwise, both $\mathcal{P}(q, r)$ and $\mathcal{P}(p, q)$ are singleton
sets so that, by the transitivity property of preorders, so is
$\mathcal{P}(p, r)$; and the composition map is again uniquely determined.
\end{itemize}

Conversely, every category $\mathcal C$ with a set of objects and
subsingleton~(\viz,~a subset of a singleton) hom-sets has an underlying
preorder structure given by
\[
  \big(\, \Ob(\mathcal C) \,,\, \leq \,\big)
\]
where $X \leq Y \iffdef ( \mathcal{C}(X, Y) \iso \nelem 1 )$.

\paragraph{Locally small and small categories.}
%
The notion of category we have considered, with a class of objects and
hom-sets, is technically referred to as \emph{locally small}.  Categories with
a set of objects are referred to as \emph{small}.  The categories associated
to monoids and preorders are small.

\subsection{Morphisms of monoids, preorders, and categories}

We recall the notions of morphism between monoids and preorders, and
generalize them to categories.

\begin{center}\begin{tabular}{ccc}
Monoids and homomorphisms & \qquad\qquad\qquad & Preorders and monotone
functions
\\[3mm]
$(M_1, \shortmid_1, \star_1)
 \stackrel{h}{\longrightarrow}
 (M_2, \shortmid_2, \star_2)$
&&
$(P_1, \leq_1) \stackrel{f}{\longrightarrow} (P_2, \leq_2)$
\\[3mm]
\begin{tabular}{l}
$h: M_1 \to M_2$
\\[1mm]
s.t.~$h(\shortmid_1) =\ \shortmid_2$
\\[1mm]
\phantom{s.t.~}$h(x \star_1 y) = h(x) \star_2 h(y)$
\end{tabular}
&&
\begin{tabular}{l}
$f: P_1 \to P_2$
\\[1mm]
s.t.~$p \leq_1 q \implies f(p) \leq_2 f(q)$
\end{tabular}
\end{tabular}
\end{center}

\begin{center}\begin{tabular}{c}
Categories and functors
\\[3mm]
$\mathcal{C}_1 \stackrel{F}{\longrightarrow} \mathcal{C}_2$
\\[4mm]
$\left\{
\begin{array}{l}
F_{\textrm{obj}} : \Ob(\mathcal{C}_1) \to \Ob(\mathcal{C}_2)
\\[3mm]
F_{\textrm{arr}} : \Arr(\mathcal{C}_1) \longrightarrow \Arr(\mathcal{C}_2)
\end{array}
\right.$
\\[8mm]
\begin{tabular}[t]{lc}
s.t. &
\\
&
% the yshift is needed to make the F_arrow level.
\begin{tikzcd}
P
\arrow[dd, "f"{name=L, right}] & & F_{obj}(P)
\arrow[dd, "F_{\text{arr}}(f)"{name=R, left}]
\\
\arrow[rr, "F_{arr}", from=L, to=R, start anchor={[yshift=-0.2ex]}, mapsto]
&  &
\\
Q &  & F_{obj}(Q)
\end{tikzcd}
\\[20mm]
&
$F_{\textrm{arr}}( \id{X} ) \,=\, \id{ F_{\textrm{obj}}(X) }$
\\[6mm]
&
\begin{tikzcd}
A \arrow[d, "f"] \arrow[dd, "g \comp f"', bend right=50]
&
F_\text{obj}(A)
\arrow[d, "F_\text{arr}(f)"{left}]
\arrow[dd, "F_\text{arr}(g \comp f)", bend left=55]
\\
B \arrow[d, "g"] & F_\text{obj}(B) \arrow[d, "F_\text{arr}(g)"{left}]
\\
C & F_\text{obj}(C)
\end{tikzcd}
\end{tabular}
\end{tabular}\end{center}

