\chapter{How to think Categorically}
\lecturedetails{5 October 2017}{M Fiore, D Makwana, S Steenkamp, O Richardson}

Introduction to Category Theory with applications to Logic and Type Theory.

\emph{Not a spectator sport.}

The approach is to find an understanding somewhere between the concrete
(examples) and the abstract (``nonsense''). The heart of category theory are
universal properties, `initial' and `final'.
Rather than giving a definition of a category straight away, we will first
explore \emph{Universal Properties} through example problems; this will
constitute the first few lectures.

\section{Examples}
We will begin with known mathematical constructions, viewed from a universal,
categorical point of view.

\subsection{Adding an element to a set}
\label{add-element-set}

Given a set $S$, consider

\begin{align*}
    \textbf{data:}&\qquad S \overset{f}{\rightarrow} S^\prime \ni x \\
    \textbf{criteria:}& \qquad
    \begin{tikzcd}[ampersand replacement=\&]
        S \arrow[r, "f_1"]
          \arrow[dr, swap, "f_2"]
          \&
        S_1 \arrow[d, "h"] \& \ni \& x_1 \arrow[d, mapsto]
          \\
          {}\&
        S_2 \& \ni \& x_2
    \end{tikzcd}
\end{align*}

The function $h : S_1 \rightarrow S_2$ must satisfy $h(x_1) = x_2$ and
$\forall x \in S .\; h(f_1(x)) = f_2(x)$.

\begin{framed}
Diagrams such as the above implicitly assume that compositions commute (that's
why they're called commutative diagrams). We can express the above more
concisely as:
\begin{center}
$h\comp f_1 = f_2$
\quad or \quad
$h\icomp f_1 = f_2$
\end{center}
where
$$
\begin{tikzcd}
    A \arrow[r, "f"] \arrow[rr, swap, bend right=40, "g \circ f"] & B \arrow[r, "g"] & C
\end{tikzcd}
$$
with $(g\comp f)(a)\eqdef g(f(a))$ for all $a\in A$.
\end{framed}

\subsubsection*{Initial (Universal) Solution}

An \emph{initial} solution is one that uniquely maps to any other solution.
Note that such a solution doesn't always exist. In the case of adding a single
element to a set, the problem can be specified in terms of finding an
appropriate $S \overset{f}{\rightarrow} T \ni t$ such that

\begin{equation*}
  \forall S \overset{f'}{\rightarrow} S' \ni x \ .\
  \exists!\ h : T \rightarrow S' \text{ such that }
  h(t) = x \text{ and }
  h \circ f = f'
\end{equation*}

We present our solution:
\begin{center}
    \begin{tikzcd}
        S \arrow[r, "\iota"]
          \arrow[dr, swap, "f"]
          &
        S_{\ast}
        \arrow[d, "h"]
        & \ni & \ast
        \arrow[d, mapsto]
          \\
        {}
          &
        T & \ni & t
    \end{tikzcd}
\end{center}

where $S_\ast$ is defined as
$\setof{ \ast } \union \bigsetof{ \lbrack x\rbrack \suchthat x \in S }$
and $\iota: S \to S_\ast$ as $\iota(x)=[x]$ for all $x\in S$.
%
(The set $S_\ast$ is an \emph{implementation} of the disjoint union of
$\setof{*}$ and $S$; the square brackets tag the elements from $S$.)

Define $h$, for $z\in S_\ast$, as
\begin{equation*}
    h(z) = \begin{cases}
        t & \text{if } z = \ast\\
        f(x) & \text{if } z = \lbrack x \rbrack
    \end{cases}
\end{equation*}

For universality we must also prove that this is a unique solution.
Consider $k:S_\ast\to T$ that satisfies $k \circ \iota = f$ and
$k(\ast) = t$. Then, we are required to prove (RTP):
$k(z) = h(z)\ \forall z \in S_\ast$.

\subsubsection*{Final solution}

There is a second kind of universal property, callled a final solution: any
other solution admits a unique map to the final one, again examining the case
of adding an element to a set, our criteria now is inverted. As we are looking
for a final solution, any $U\ni u$ must admit a unique map to our solution
$T\ni t$, as shown below.
\begin{center}
    \begin{tikzcd}
        S \arrow[r, "f"]
          \arrow[dr, swap, "\forall g"]
          &
        T & \ni & t
          \\
        {}
          &
          U
          \arrow[u, swap, "\exists! h"]
          & \ni & u
          \arrow[u, swap, mapsto]
    \end{tikzcd}
\end{center}

For a set $A$, let $!_A$ be the unique function $A\to \setof{\ast}$ given by
the constantly $\ast$ function, that is, $!_A(a) = \ast$ for all $a\in A$.

Then, $T=\setof{\ast}$, $t=\ast$, and $f=\ !_S$ is a final solution.

(Note the correspondence between $\ast \in \{ \ast \}$ and \texttt{():unit}
in most functional programming languages.)

\subsection{Adding two sets}

Given $A$ and $B$ sets.

\begin{align*}
    \textbf{data:} \qquad& \begin{tikzcd}[ampersand replacement=\&]
            A \arrow[r, "f"] \& C \& B \arrow[l, swap, "g"]
        \end{tikzcd}\\
    \textbf{criteria:} \qquad& \begin{tikzcd}[ampersand replacement=\&]
        {} \& C_1 \arrow[dd, "h"] \& {} \\
        A \arrow[ru, "f_1"] \arrow[rd, swap, "f_2"]
        \& {}
        \&
        B \arrow[lu, swap, "g_1"] \arrow[ld, "g_2"]
        \\
        {} \& C_2 \& {}
    \end{tikzcd}
\end{align*}

That is, such that (1)~$h$ is a function $C_1 \rightarrow C_2$ and (2)~it
makes the diagram commute~(\ie~$h \circ f_1 = f_2$ and $h \circ g_1 = g_2$).

\begin{exercise}
    Find the initial solution. \label{ex:addsets}
\end{exercise}

\subsection{Duality}

When you have a problem in ``one direction'' you can turn it around and get a
problem in ``the other direction''. This involves flipping the directions of
the arrows in the commutative diagrams.

\subsubsection*{Example: Dual form of adding two sets}

Given two sets $A$ and $B$.

\begin{align*}
     \textbf{data:}\qquad & \begin{tikzcd}[ampersand replacement=\&]
            A \& C \arrow[l, swap, "f"] \arrow[r, "g"] \& B
        \end{tikzcd} \\
    \textbf{criteria:}\qquad & \begin{tikzcd}[ampersand replacement=\&]
          {} \& C_1 \arrow[dd, "h"] \arrow[ld, swap, "f_1"] \arrow[rd, "g_1"]
          \& {} \\ A \& {} \& B \\
          {} \& C_2 \arrow[lu, "f_2"] \arrow[ru, swap, "g_2"] \& {}
      \end{tikzcd}
\end{align*}



That is, such that (1)~$h$ is a function $C_1 \rightarrow C_2$ and (2)~it
makes the diagram commute~(\ie~$h \circ f_2 = f_1$ and $h \circ g_2 = g_1$).

\begin{exercise}
    Find the final solution. \label{ex:dual}
\end{exercise}

%%% MF ???
%It is worth noting that a dual solution can be easier to solve and allow
%insight into the original problem, or vice versa.

\subsection{Generating a monoid}
\label{section_generate_monoid}

\subsubsection*{Background on Monoids}

A \emph{monoid} is a triple $(M, e, \ast)$ where $M$ is a set, $e \in M$ is a
neutral element, and $\ast$ is a binary operation on $M$; where $e$ and $\ast$
obey:
\begin{align*}
    & \forall x \in M\ .\
    e \ast x = x \text{ and } x \ast e = x && \text{(neutral
    element)} \\
    & \forall x,y,z \in M\ .\
    (x \ast y) \ast z = x \ast (y \ast z) && \text{(associativity)}
\end{align*}

\subsubsection*{Monoid homomorphism}

A monoid homomorphism is a function between (the underlying sets of) two
monoids that preserves the unit and the multiplication. Formally,
$h : (M_1, e_1, \ast_1) \rightarrow (M_2, e_2, \ast_2)$ is a function
$h: M_1\to M_2$ such that $h(e_1) = e_2$ and
$\forall x, y \in M_1\ .\ h(x \ast_1 y) = h(x) \ast_2 h(y)$.

Our next exercise will deal with the \emph{free monoid} on a set.
Given a set $S$, we would like to construct a monoid. In the language we've
been using, we can present it as follows:

\begin{align*}
    \textbf{data:}\qquad& S \overset{f_1}{\longrightarrow}M
    \qquad \text{$(M,e,\ast)$ a monoid}\\
    \textbf{criteria:}\qquad& \begin{tikzcd}[ampersand replacement=\&]
        S \arrow[r, "f_1"] \arrow[dr, swap, "f_2"] \&
        M_1
        \arrow[d, "h"]
        \& (M_1,e_1,\ast_1)
        \arrow[d, "\text{$h$ an homomorphism}"]
        \& \text{a monoid}
        \\
        {}
        \& M_2 \& (M_2,e_2,\ast_2) \& \text{a monoid}
    \end{tikzcd}
\end{align*}

\begin{exercise}
    Find the initial solution in this case.
\end{exercise}
