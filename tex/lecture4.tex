\chapter{Categories}
\lecturedetails{17 October 2017}{M Fiore, B Hardy}

\section{Definitions}

\begin{definition} Categories (internal viewpoint)

  A category consists of two kinds of entities:
  \begin{itemize}
  \item Objects
  \item Morphisms between objects (also called maps or arrows)
  \end{itemize}

  Every morphism comes with an assigned domain (also source) and codomain (also
  target). We write $D \morpharr f C$ to show that the morphism $f$ has domain
  $D$ and codomain $C$.

  Every object $A$ has an associated \emph{identity} arrow $A \morpharr{\id A}
  A$, also denoted $1_A$ or just $A$.

  Arrows come with a (well-typed) \emph{composition} operation $\circ$. Given
  $A \morpharr f B$ and $B \morpharr g C$ we have $A \morpharr{g \circ f} C$,
  also denoted $A \morpharr{gf} B$ or $A \morpharr{f;g} B$ (read ``apply $f$ and
  then $g$'').

  Composition and identity arrows must satisfy laws similar to the monoid laws:

  \begin{itemize}
  \item Composition is associative. Given any objects and arrows as follows

    \[A \morpharr h B \morpharr g C \morpharr f D\]

    we require that

    \[(f \circ g) \circ h = f \circ (g \circ h).\]

  \item $\id X$ is an identity under composition. Given any

    \[X \morpharr k Y\]

    we require that

    \[k \circ \id X = k = \id Y \circ k .\]
  \end{itemize}
\end{definition}

\begin{definition} Categories (enriched viewpoint)

  A category $\mathcal C$ consists of
  \begin{itemize}
  \item a class of objects $\Ob(\mathcal C)$ (also written $|\mathcal C|$),
    together with
  \item \emph{hom-sets} $\mathcal C(A, B)$ for all $A, B \in \Ob(\mathcal C)$,
    and
  \item identities $\id A \in \mathcal C(A, A)$ for all $A \in \Ob(\mathcal C)$,
    in addition to
  \item a composition operation
    \begin{equation*}
      \begin{matrix}
        \mathcal C(B, C) & \times & \mathcal C(A, B) &
        \morpharr{\circ_{A,B,C}}
        & \mathcal C(A, C) \\
        g & , & f & \mapsto & g \circ f
      \end{matrix}
    \end{equation*}
  \end{itemize}

  It must satisfy laws, written here as commutative diagrams:

  \begin{itemize}
  \item Composition is associative:

    \begin{center}
    \begin{tikzcd}
      & \mathcal C(C, D) \times \mathcal C(B, C) \times \mathcal C(A, B)
      \arrow[dl, "\circ_{B,C,D} \times \idfunc", swap]
      \arrow[dr, "\idfunc \times \circ_{A,B,C}"]
      & \\
      \mathcal C(B, D) \times \mathcal C(A, B) \arrow[dr, "\circ_{A,B,C}",
      swap] & &
      \mathcal C(C, D) \times \mathcal C(A, C)
      \arrow[dl, "\circ_{A,C,D}"] \\
      & \mathcal C(A, D) &
    \end{tikzcd}
  \end{center}

  \begin{framed}
    \paragraph{NB.} Here, $\idfunc \eqdef (x \mapsto x)$ is the standard
    identity function.  The operator $\times$ on two functions intuitively runs
    two them in parallel.
    Given

      \[f : A_1 \rightarrow B_1
        \enspace,\quad
        g : A_2 \rightarrow B_2\]

      we have
      \[f \times g : A_1 \times A_2 \rightarrow B_1 \times B_2\]
      defined by
      \[\big(f \times g\big)(x, y) \mapsto (f x, g y) \]
  \end{framed}

  \item Identity is a left and right neutral element. First, notice that we
    can write

    \[\nelem{1} \morpharr{\id A} \mathcal C(A, A)\]

    to express the idea that the identity morphism $\id A$ exists for every
    $A$.  Then the right identity law is expressed as:

    \begin{center}
    \begin{tikzcd}
      \mathcal C(A, B) \times \nelem{1} \arrow[rr, "\idfunc \times \id A"]
      \arrow[dr, "\pi_1", swap] & &
      \mathcal C(A, B) \times \mathcal C(A, A) \arrow[dl, "\circ_{A,A,B}"] \\
      & \mathcal C(A, B)
    \end{tikzcd}
  \end{center}

    and the left identity law is expressed similarly as:

    \begin{center}
    \begin{tikzcd}
      \nelem{1} \times \mathcal C(A, B) \arrow[rr, "\id B \times \idfunc"]
      \arrow[dr, "\pi_2", swap] & &
      \mathcal C(B, B) \times \mathcal C(A, B) \arrow[dl, "\circ_{A,B,B}"] \\
      & \mathcal C(A, B)
    \end{tikzcd}
  \end{center}
\end{itemize}

\end{definition}

\section{Examples of categories}

\subsection{Sets}

$\catset$ is a category whose objects are sets and whose morphisms are standard
set-theoretic functions. The identity is the standard identity function and
composition is the standard composition of functions. The set of morphisms can
be written as

\[\catset(S, T) = \setof{ f \subseteq S \times T \;|\; f \text{ is a function}
  }\]

\subsection{Monoids}

$\catmon$ is a category whose objects are monoids $(M,\shortmid,\star)$ and
whose morphisms are monoid homomorphisms (\ie~functions which preserve the
monoid structure).

Identity arrows are the usual identity functions and composition is the usual
function composition.

We can check that composition of monoid homomorphisms always produces a monoid
homomorphism. Assume we have monoids
\[
  (M_1, \shortmid_1, \star_1)
  \enspace,\quad
  (M_2, \shortmid_2, \star_2)
  \enspace,\quad
  (M_3, \shortmid_3, \star_3)
  \enspace.
\]
and let $f : M_1 \rightarrow M_2$ and $g : M_2 \rightarrow M_3$ be monoid
homomorphisms.

Then,
\[
  \big(g\comp f\big)(\shortmid_1)
  \ = \ g\big(f(\shortmid_1)\big)
  \ = \ g(\shortmid_2)
  \ = \ \shortmid_3
\]
and, for $x, y \in M_1$,
\begin{align*}
  \big(g \circ f\big)(x \star_1 y)
  & = g \big(f (x \star_1 y)\big)
    & \text{(definition of function composition)}\\
  & = g \big(f (x) \star_2 f (y)\big) & \text{($f$ is a monoid homomorphism)}\\
  & = g\big(f(x)\big) \star_3 g\big(f(y)\big)
    & \text{($g$ is a monoid homomorphism)}\\
  & = \big(g \circ f\big)(x) \star_3 \big(g \circ f\big)(y)
    & \text{(definition of function composition)}
\end{align*}

\subsection{Preorders}

$\catpreo$ is a category whose objects are preorders $(P, \le_P)$ and whose
morphisms are monotone functions.

For preorders $(P, \le_P)$ and $(Q, \le_Q)$, a function
\[f : P \rightarrow Q\]
is \emph{monotone} when, for every $x, y \in P$,
$x \le_P y \implies f(x) \le_Q
f(y)$.

In this category, $\id{}$ and $\circ$ are as in $\catset$.

\subsection{Partial Orders}

$\catposet$ is a category whose objects are partial orders and whose morphisms
are monotone functions.

Notice that $\catposet$ is a \emph{subcategory} of $\catpreo$.

\begin{framed}
  \begin{definition} Subcategory

    A subcategory $\mathcal S$ of a category $\mathcal C$ is specified by
    \begin{itemize}
    \item $\Ob(\mathcal S) \subseteq \Ob(\mathcal C)$ and
    \item $\Arr(\mathcal S) \subseteq \Arr(\mathcal C)$
    \end{itemize}
    with identities and composition in $\mathcal S$ as in $\mathcal C$. $\Ob$ is
    defined as above, and $\Arr$ is the set of ``arrows'' or morphisms in
    $\mathcal S$.
  \end{definition}
\end{framed}

\subsection{Finite sets}

$\catfinset$ is the subcategory of $\catset$ whose objects are \emph{finite}
sets.

\subsection{Bijections}

$\catbij$ is the subcategory of $\catfinset$ whose morphisms are all bijections.

Notice that, relative to $\catset$, $\catfinset$ cut down the objects but kept
the morphisms; and, relative to $\catfinset$, $\catbij$ cut down morphisms but
kept the objects.

\subsection{Relations}

$\catrel$ is a category whose objects are sets and whose arrows are relations
between sets:

\[\catrel(A, B) = \setof{\, R \;|\; R \subset A \times B \,}\]

Identities are given by the identity relations, relating every element to
itself:

\[\id A = \setof{\, (a, a) \;|\; a \in A \,}\]

and composition is relational composition:

\[
  R \circ_{A,B,C} S
  =
  \bigsetof{\,
    (a, c) \;|\; \exists b \in B.\; (a, b) \in S \wedge (b, c) \in R
  \,}
\enspace.
\]

Notice that $\catset$ is a subcategory of $\catrel$. In fact, there is an
intermediate category $\catpfn$ whose morphisms are \emph{partial} functions.

\subsection{Predicates}

$\catpred$ is the category whose objects are pairs $(S, P)$ of a set $S$ and a
predicate on $S$, $P \subseteq S$. Its morphisms are functions that respect
the predicates, \ie

\[(S_1, P_1) \morpharr{f} (S_2, P_2)\]

is a function $f : S_1 \rightarrow S_2$ such that

\[\forall x \in S.\; x \in P_1 \implies f(x) \in P_2 .\]

Identity and composition are as for functions.

\section{Exercises}

Recall the following universal property from Lecture~1:

\begin{center}
\begin{tikzcd}
    {} & P \arrow[ld, swap, "\alpha"] \arrow[rd, "\beta"] & {} \\
    A
    & {}
    &
    B
    \\
    {} &
    \stackrel{\mbox{$C$}}{\forall}
    \arrow[uu, dashed, "\exists!"]
    \arrow[lu, "f"] \arrow[ru, swap, "g"] & {}
\end{tikzcd}
\end{center}

Before, we stipulated that $P, A, B, C$ were sets and $\alpha, \beta, f, g$
were functions. However, we can generalize those assertions and instead work
in an arbitrary category $\mathcal C$; so that $P, A, B, C$ are objects of
$\mathcal C$ and $\alpha, \beta, f, g$ are arrows of $\mathcal C$ between the
respective objects.

\begin{exercise}
  In $\catset$, we can take
  $P = A \times B\eqdef \bigsetof{\, (a,b) \suchthat a\in A \wedge b\in B \,}$,
  $\alpha = \pi_1: (a,b)\mapsto a$ and $\beta = \pi_2: (a,b)\mapsto b$ as the
  final solution.  Check it out!

  What happens for the other categories described in this lecture? (Note that in
  some of the categories there will be no solution.) In particular, have a look
  at $\catmon$, $\catpreo$, $\catpred$, and $\catrel$.
\end{exercise}
