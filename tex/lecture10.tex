\chapter{Representability}
\lecturedetails{7 November 2017}{M Fiore, P Fernandes, P Bose}

\begin{definition}[Representation]
Consider a category $\mathcal{C}$ and a functor 
$K: \mathcal{C}^{op} \rightarrow \catset $.
A representation is of $K$ is then given by:
\begin{itemize}
  \item an object $R \in \mathcal{C}$
  \item a natural isomorphism $\rho: \mathcal{C(\phold, R)} \natarrow K(\_) $
  \end{itemize}
\end{definition}
Notice that representations are unique up to isomorphism since:
\[\mathcal{C}(\phold, R_1) \cong K(\phold) \cong 
\mathcal{C}(\phold, R_2)\ \iff R_1 \cong R_2\]

\begin{example}
If we consider $K_{A,B}(\phold) = \mathcal{C}(\phold, A) \times \mathcal{C}
(\phold, B)$, we have from last chapter lecture that $A \times B$ coupled with 
$\rho : \{ (\pi_1 \circ \phold, \pi_2 \circ \phold) \}$ is a representation.
\end{example}

\begin{example}
Consider  $K: \mathcal{C}^{op} \rightarrow \catset $ such that 
$K(c) = \mathbb{1}$. Then we have that:
\[\mathcal{C}(\phold, R) \cong K(\phold) = \mathbb{1}\]
So here the objects of the representation are terminal objects.

\end{example}
\section{Yoneda Lemma}
\begin{theorem}[Yonneda Lemma]
There is a natural bijection correspondence 
\[ 
  \Efrac
    {\mathcal{C}(\phold, A) \natarrow K}
    {K(R) }
\]
\end{theorem}
\begin{exercise}
Prove the Yonneda Lemma
\end{exercise}
\begin{example}
Consider the category:\\
\[\mathcal{W}: (0 \longleftarrow 1 \longleftarrow 2 \longleftarrow 3 
\longleftarrow ...) \]
We then have that a $K_{\mathcal{W}}$ must be of the form
\[K_{\mathcal{W}}: (K_{\mathcal{W}}(1) \longrightarrow K_{\mathcal{W}}(2) 
\longrightarrow K_{\mathcal{W}}(3)  \longrightarrow ...) \]
And that $\mathcal{W}(\phold, n)$ must be
\[
\mathcal{W}(\phold, n):
\begin{tikzcd}
\mathcal{W}(0, n) \arrow[r] & \mathcal{W}(1, n) \arrow[r] & ... 
\arrow[r] & \mathcal{W}(n, n) \arrow[r] & ... \\
\emptyset \arrow[u, phantom, "\rotatebox{90}{$=$}"] & \emptyset 
\arrow[u, phantom, "\rotatebox{90}{$=$}"] & ... & 
\nelem{1} \arrow[u, phantom, "\rotatebox{90}{$=$}"] & ...
\end{tikzcd}
\]
Look natural transformations between $\mathcal{W}(\phold, n)$ 
and K, so transformations such that:
\begin{center}
\begin{tikzcd}
\emptyset \arrow[r] \arrow[d] & \emptyset \arrow[r] \arrow[d] & ... \arrow[r]
 & \nelem{1} \arrow[r] \arrow[d] & ...\\
K_{\mathcal{W}}(0) \arrow[r] & K_{\mathcal{W}}(1) \arrow[r] & ... \arrow[r] 
& K_{\mathcal{W}}(n) \arrow[r] & ...
\end{tikzcd}\\
commutes
\end{center}
Then to give a natural transformation $\phi : \mathcal{C}(\phold, A) 
\natarrow K$ is the same as giving an element of $K_\mathcal{W}(n)$
\end{example}
\begin{definition}[Yoneda Embedding]
Recall $\mathcal{C}(\phold, \phold): \mathcal{C}^{op} \times \mathcal{C}
 \rightarrow \catset$.\\
By "currying", we have $y: \mathcal{C} \rightarrow \catset^{\mathcal{C}^{op}}$
 such that for all arrows $f: A \rightarrow B$ in $\mathcal{C}$:
\begin{center}
\begin{tikzcd}
A \arrow[mapsto]{r} \arrow{dd} & \mathcal{C}(-,A) \arrow{dd} \\
f \ \ \ \arrow[mapsto]{r} & \ \ \ (f \circ -) \\
B \arrow[mapsto]{r} & \mathcal{C}(-,B) \\
\end{tikzcd}
\end{center}

Note $y_{A,B} : \mathcal{C}(A,B) \rightarrow \catset^{\mathcal{C}^{op}}(y_A, y_B)$ 
is a bijection, and thus defines the isomorphism:
\[ \mathcal{C}(A,B)  \cong  Nat( \mathcal{C}(-,A), \mathcal{C}(-,B)) \]

Intuitively $\catset^{\mathcal{C}^{op}}$ has an isomorphic copy of 
$\mathcal{C}$ in it. Such functors are called embeddings.
\end{definition}


\begin{definition}
Let $F : \mathcal{A} \rightarrow \mathcal{B}$ be a functor. We say $F$ is 
faithful if $F_{X,Y} : \mathcal{A}(X,Y) \rightarrow \mathcal{B}(F(X),F(Y))$ 
is an injection for all objects $X,Y$ in $\mathcal{A}$. We say $F$ is full 
if $F_{X,Y}$ is surjective for all objects $X,Y$ in $\mathcal{A}$.
\end{definition}


\section{Representing Homs}

Can we represent/internalise the set of arrows $\mathcal{C}(A,B)$ as an object in 
$\mathcal{C}$, say $(A \Rightarrow B)$?

\begin{definition}[Exponentials]
The exponential of objects $A$ and $B$ in a cartesian category $\mathcal{C}$ 
is a representation of parameterised maps from $A$ to $B$, that is:
\[  \mathcal{C}(-, A \Rightarrow B) \cong \mathcal{C}(-\times A, B)  \]

\end{definition}

Equivalently an exponential is given by an evaluation map $\epsilon_{A,B} 
\in \mathcal{C} ((A \Rightarrow B) \times A, B)$ such that for all objects $X$ 
and arrows $f: X \times A $ in $\mathcal{C}$, there exists a unique arrow 
$\lambda^{A,B}(f) : X \rightarrow (A \Rightarrow B)$ making the following diagram commute:
\begin{center}
\begin{tikzcd}
(A \Rightarrow B) \times A \arrow[r, "\epsilon_{A,B}"] & B \\
X \times A \arrow[u, "\lambda^{A,B}(f) \times Id_A"] \arrow[ur, "f"] & {} 
\end{tikzcd}
\end{center}


\textbf{Application}: Suppose $\mathcal{C}$ is a bicartesian (has products and 
coproducts) and closed (has exponentials). Then $\mathcal{C}$ is distributive,
 that is the natural map:
\[  \delta: (X \times A) + (X \times B) \rightarrow X \times (A + B )\]
is an isomorphism.

\begin{exercise}
Build the inverse of $\delta$ above. Recall:

\[ P \cong Q \ \ \  \text{ iff }  \Efrac{Z \rightarrow P}{Z \rightarrow Q} 
\text{natural in Z} \ \ \ \text{iff}     \Efrac{P \rightarrow Z}{Q \rightarrow Z} \]
\end{exercise}