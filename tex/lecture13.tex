\chapter{Adjoint Functors}
\lecturedetails{16 November 2017}{M Fiore, R Kusztos}

\section{More about subobjects classifiers}

Recall the definition of the \underline{Sub} functor: 

\begin{align*}
  \underline{Sub}: \lscat{C}^{op} &\longrightarrow \textbf{Set} \\
  X &\longmapsto \underline{Sub}(X) = \bigl\{ [m]_{\approx} | 
    \textrm{m is mono into X} \bigl\} \\ 
\end{align*}

\begin{center}
  \begin{tikzcd}
    f^{*}(P)
      \arrow[phantom, "\lrcorner", pos=0]{dr} 
      \arrow[r] 
      \arrow[d] & 
    X 
      \arrow[d] \\
    P
      \arrow[r] & 
    Y
  \end{tikzcd}
\end{center}

What is a representation for this?
\begin{definition}[Subobject classifiers]
  To give a natural isomorphism, 
  $\lscat{C}(\_,\Omega) \overset{\iso}{\Longrightarrow} \underline{Sub}(\_) $ 
  is equivalent to giving an 
  $\Omega \in \lscat{C}$ 
  and a 
  $ \Bigl[t \underset{\Omega}{\overset{T}{\downarrow}} \Bigl] 
    \in \underline{Sub}(\Omega)$
  such that for all 
  $X \in \lscat{C}^{\Omega}$
  there exists a bijective correspondence: 

  \begin{align*}
    \forall 
    \Bigl[m \underset{X}{\overset{P}{\downarrow}} \Bigl] 
    \in \underline{Sub}(X) \quad
    \exists! 
    X \overset{\chi_{[m]}}{\longrightarrow} \Omega \quad 
    \textrm{such that} \quad
    \chi_{[m]}^{*}[t] = [m]
  \end{align*}
\end{definition}

\begin{definition}[Subobject classifiers]
  \label{subobject_classifier_def}
  The previous definition is equivalent to simply giving 
  $\Omega \in \lscat{C}$
  and 
  $t \underset{\Omega}{\overset{T}{\downarrow}} \in \lscat{C}$
  such that 
  \begin{align*}
    \forall 
    m \underset{X}{\overset{P}{\downarrow}}
    \in \underline{Sub}(X) \quad
    \exists! 
    X \overset{\chi_{m}}{\longrightarrow} \Omega
  \end{align*}  
  \begin{center}
    \begin{tikzcd}
    P 
      \arrow[phantom, "\lrcorner", pos=0]{dr} 
      \arrow[r, "u"] 
      \arrow[d, "m"] & 
    T 
      \arrow[d, "t"] \\
    X
      \arrow[r, "\chi_m"] & 
    \Omega
    \end{tikzcd}
    \end{center}
  for some $u$
\end{definition}

\begin{example}[in $\textbf{Set}$]
\begin{align*}
  1 \overset{t}{\longrightarrow} \{t, f\} = \Omega
\end{align*}
\begin{align*}
  Sub(X) = \mathcal{P}(X)
\end{align*}

\begin{center}
  \begin{tikzcd}
    P 
      \arrow[phantom, "\lrcorner", pos=0]{dr} 
      \arrow[r, "u"] 
      \arrow[d, "m"] & 
    1
      \arrow[d, "t"] \\
    X
      \arrow[r, "\exists!\varphi"] & 
    \{t, f\}
    \end{tikzcd}
  \end{center}

Where $\varphi$ is such that $\varphi^{-1}\{t\} = P$ \\
Since we are in set, $\varphi$ is the characteristic function of X

\begin{equation*}
 \varphi = \chi_m(x) = 
  \begin{cases}
    t & x \in P \\
    f & x \notin P 
  \end{cases}
\end{equation*}

\end{example}

\begin{exercise}
  In the subobject classifier definition 
  (Definition \ref{subobject_classifier_def}), $T$ is necessarily a
terminal object
\end{exercise}

\begin{definition}[Toposes]
  A topos is a category with: 
  \begin{itemize}
    \item all finite limits 
    \item exponentials
    \item subobject classifier $\Omega$
  \end{itemize}

Examples of toposes are: 
  \textbf{Set}, 
  $\textbf{Set}^{\mathbb{C}^{op}}$, 
  \textbf{DirGraph}

\end{definition}

\begin{remark}
  Toposes are models of Higher Intuitionistic Logic.
\end{remark}

\begin{exercise}
  What is $\Omega$ in \textbf{DirGraph}?

  For some graph 
  $G = (N, E)$, 
  Sub(G) = the set of all its subgraphs, each given by a subset of nodes,
  $S_n \subseteq N$
  and a subset of edges 
  $E_n \subseteq E$ 
  between nodes in $S_n$.

  \begin{center}
    \begin{tikzcd}
      \bullet 
        \arrow[rr, bend left] & {} \arrow[dd, hook'] & 
        \bullet \arrow[loop, looseness=4] &  &  &  & 
        \bullet \arrow[dd, hook'] \arrow[loop, looseness=4] \\
       &  &  &  &  &  &  \\
       & {} &  &  &  &  & {} \\
      \bullet \arrow[rr, bend left] 
      \arrow[rd, bend right] &  & 
      \bullet
        & {} 
      \arrow[rr] &  & {} & 
      \textcolor{green}{\bullet} 
      \arrow[loop left, green]
      \arrow[loop right, red]
      \arrow[bend left=20, blue]{d} \\
       & \bullet \arrow[ru, bend right] \arrow[lu, bend right] &  &  &  &  & 
       \textcolor{red}{\bullet} 
       \arrow[loop right, red]
       \arrow[bend left=20, blue]{u}
    \end{tikzcd}
  \end{center}
\end{exercise}

\begin{exercise}
  What is $\Omega$ in $\textbf{Set}^{(0 \to 1)}$, also known as the 
Sierpinski topos.
\end{exercise}

\section{Adjoint Functors}

\begin{example}
  In the first lecture we introduced the free functor which maps a set to 
the free monoid:

  \begin{align*}
    \textbf{Set} &\overset{F}{\longrightarrow} \textbf{Mon} \\
    S &\longmapsto (S^*, t, \circ)
  \end{align*}
  Similarly, we can introduce the forgetful functor U, which gives the set 
underlying a monoid.

\begin{center}
  \begin{tikzcd}[sep=3em]
    \textbf{Mon}
      \ar["U"{right}]{d} 
    \\
    \textbf{Set}
      \ar[bend left=40, "F"{left}]{u}
  \end{tikzcd}\\[1mm]
\end{center}

Here, we observe that:

\begin{align*}
  \textbf{Mon}(F(S), \underline{M}) &\cong_{nat} 
    \textbf{Set}(S, U(\underline(M))) \\
    \shortintertext{We write that:} \\
    F &\dashv U \\
    \intertext{To say that F and U are adjoints; F is called the left adjoint
    whereas U is called the right adjoint}
  \end{align*}
\end{example}

\begin{example}[Colimits]
  \begin{align*}
    \lscat{C} &\underset{\Delta}{\longrightarrow} \lscat{C}^\mathbb{G} \\ 
    X &\longmapsto \Delta(X): \mathbb{C} \to \lscat{G} \quad 
    \textrm{which maps} \quad
    e \overset{m}{\underset{n}{\downarrow}} \longmapsto
    \overset{x}{\underset{x}{\downarrow}}id
  \end{align*} 
That is, 
  \begin{align*}
    \textrm{For} \quad D \in \lscat{C}^\mathbb{G},  \quad
    &\varphi: D \Rightarrow \Delta(X) \\
    &\equiv \textrm{cocone with vertex x for diagram D} \\
    \lscat{C}(\underline{colim}D, X) &\cong 
      \lscat{C}^{\mathbb{G}}(D, \Delta(X)) \\ 
    \underline{colim} &\dashv \Delta \\ 
    \shortintertext{Dually,} \\ 
    \lscat{C}(X, \underline{lim}D) &\cong
      \lscat{C}^{\mathbb{G}}(\Delta(X), D) \\
    \Delta &\dashv \underline{lim}  
  \end{align*}

\end{example}

\begin{definition}[Adjoint Pair]
An adojoint pair of functors, $F \dashv G: \lscat{D} \to \lscat{C}$ consists
of: 
 \begin{center}
  \begin{tikzcd}
    \lscat{D} 
    \ar[bend left=20, "G"]{r} &
   \lscat{C} 
    \ar[bend left=20, "F"]{l}
  \end{tikzcd}
\end{center}
together with a natural isomorphism:
\begin{align*}
\lscat{D}(F C, D) &\cong 
  \lscat{C}(C, G D) \\ 
\end{align*}

Equivalently, to give a left adjoint $F$ to a functor 
$G: \lscat{D} \to \lscat{G}$
is to give for all $X \in \lscat{C}$
and object $F(x) \in \lscat{D}$
together with a map $X \underset{\eta_X}{\rightarrow} G(F(X)$
in $\lscat{C} $
such that: 
\begin{center}
    \begin{tikzcd}[ampersand replacement=\&]
    X \arrow[r, "\eta_X"]
    \arrow[dr, "\forall f"{left}] \&
    G(F(X)) \arrow[d, "G(f^\#)"] \& 
    F(X) \arrow[d, mapsto, "\exists! f^\#"] \\
    {} \& G(Y) \& Y
  \end{tikzcd} 
\end{center}
\end{definition}
\begin{proposition}
  Left adjoints preserve colimits and dually, right adjoints preserve limits.
\end{proposition}

\begin{example}
A functor $F: \lscat{C} \to \lscat{D}$ preserves sums if:

  \begin{center}
    \begin{tabular}[t]{lc}
    $\forall$ sums in $\lscat{C}$
    \begin{tikzcd}
      {} & A + B & {} \\
      A \ar["\iota_1"]{ru} & & B \ar["\iota_2"above]{lu} \\
    \end{tikzcd}
    ,
    \begin{tikzcd}
      {} & F(A + B) & {} \\
      F(A) \ar["F(\iota_1)"]{ru} & & F(B) \ar["F(\iota_2)"above]{lu} \\
    \end{tikzcd}
    is a sum in $\lscat{D}$.
    \end{tabular}
  \end{center}

If $\lscat{D}$ has sums, 

\begin{center}
  \begin{tikzcd}
    & F(A) + F(B) \ar["\cong", dotted]{d} & \\ 
    & F(A + B) & \\
    F(A) \ar["F(\iota_1)"]{ru} 
         \ar["\iota^{F(A), F(B)}_1", bend left=40]{ruu} & & 
    F(B) \ar["F(\iota_2)"]{lu}
         \ar["\iota^{F(A), F(B)}_2", bend right=40]{luu} \\
    

  \end{tikzcd}
\end{center} 

We need to show that $F(A) + F(B) \cong F(A + B)$. We have previously shown
that this is equivalent to showing that: 

\[ 
  \Efrac
    {F(A) + F(B) \to z}
    {F(A + B) \to z}
\]

Intuitively,

\LARGE\[
  \Efrac
  {
  \Efrac
    {F(A) + F(B) \to z} 
    { \Efrac 
      {F(A) \to z}  
      {A \to G z}
      \quad 
      \Efrac 
      {F(B) \to z}
      {B \to G z}
    }
  }
  {
    \Efrac 
    {A + B \to G z}
    {F(A + B) \to z}
  }
\]

\end{example}

\begin{example}
  Exponentials are adjoints.
  \begin{align*}
    \_ \times A &\dashv A \Rightarrow (\_) \\
    \lscat{C}(X (\times A), B) &\cong \lscat{C}(X, (A \Rightarrow) B)
  \end{align*}
\end{example} 
