\chapter{Limits}
\lecturedetails{14 November 2017}{M Fiore, N Licker, S Borgeaud}

\section{Products}

Recall that in a category with products, we have a $\prod_{i \in I} X_i$
together with 
\begin{center}
  \begin{tikzcd}
    \prod_{i \in I} X_i
      \arrow["\pi_k"]{d}
    \\
    X_k
  \end{tikzcd}
\end{center}
for all $k \in I$.

\section{Equalizers}

An \emph{equalizer} of a parallel pair $f,g: A\to B$ is an object $E$ and a
morphism $e: E \to A$ such that $fe=ge$ and for all objects $X$ and morphisms
$x: X \to A$ satisfying $fx=gx$, there is a unique $u: X \to E$ such that
$eu = x$.

\begin{center}
  \begin{tikzcd}
    E
      \arrow["e"]{r}
    &
    A
      \arrow["f"above, shift left=+1]{r}
      \arrow["g"below, shift left=-1]{r}
    &
    B
    \\
    &
    \stackrel{\mbox{$X$}}\forall 
      \arrow["\forall x",swap]{u}
      \arrow["\exists! u", dotted]{ul}
    &
  \end{tikzcd}
\end{center}

\begin{example}
In $\catset$, $E = \bigsetof{ a \in A \mid f(a) = g(a) }\hookrightarrow A$ is
an equalizer of $f,g:A\to B$. 
\end{example}

\section{Pullbacks}

Pullbacks are limits of diagrams of the form:
\begin{center}
  \begin{tikzcd}
    &
    B
      \arrow["g"]{d}
    \\
    A
      \arrow["f"]{r}
    &
    C
  \end{tikzcd}
\end{center}
A span $A\stackrel p\longleftarrow P\stackrel q\longrightarrow B$ is a
\emph{pullback} of the cospan $(f:A \to C\leftarrow B: g)$ if for all spans
$A\stackrel u\longleftarrow X\stackrel v\longrightarrow B$ such that $fu=gv$,
there is a unique $w:X \to P$ such that $pw=u$ and $qw=v$:
\begin{center}
  \begin{tikzcd}
    X
      \arrow["\exists!w", dotted]{dr}
      \arrow["v", bend left=20]{drr}
      \arrow["u", bend right=20]{ddr}
    &
    &
    \\
    &
    P
      \arrow["q"]{r}
      \arrow["p"]{d}
      \arrow[phantom, "\lrcorner", pos=0]{dr}
    &
    B
      \arrow["g"]{d}
    \\
    &
    A
      \arrow["f"]{r}
    &
    C
  \end{tikzcd}
\end{center}

\begin{example}
In $\catset$, $P = \bigsetof{(a, b) \in A \times B \mid f(a) = g(b) }$,
together with the projections to $A$ and $B$, is a pullback of
$(f:A \to C\leftarrow B: g)$.
\end{example}

\begin{lemma}[Pullback Lemma]
For a commutative diagram
\begin{center}
\begin{tikzcd}
  \bullet \arrow[d] \arrow[r] & \bullet 
  \arrow[phantom, "\lrcorner", pos=0]{dr} \arrow[r] \arrow[d] & \bullet 
  \arrow[d] 
  \\ 
  \bullet \arrow[r] & \bullet \arrow[r] & \bullet 
\end{tikzcd}
\end{center}
where the right square is a pullback, the left square is a pullback if and
only iff so is the outer rectangle.
\end{lemma}

\begin{exercise}
	Proof the Pullback Lemma.
\end{exercise}


\section{Limits of diagrams}

\begin{definition}
A \emph{diagram} of shape a small category $\mathbb{G}$ in a category
$\mathcal{C}$ is a functor $\mathbb{G} \xrightarrow{D} \mathcal{C}$.
\end{definition}

\begin{example}
Diagrams for a discrete shape $\mathbb{G}$ (\ie~categories with only identity
arrows) amount to families of objects $\setof{  D_n }_{n \in \mathbb{G}}$ and
provide diagrams for [co]products.
\end{example}

\begin{example}
Diagrams for equalisers are of shape
\begin{center}\begin{tabular}{|c|}\hline
  \begin{tikzcd}
  \bullet
    \arrow[shift left=+1.5]{r}
    \arrow[shift left=-1.5]{r}
  &
  \bullet
  \end{tikzcd}\\ \hline
  \end{tabular}\end{center}
\end{example}

\begin{example}
Diagrams for pullbacks are of shape
  \begin{center}\begin{tabular}{|c|}\hline
    \begin{tikzcd}
      &
      \bullet
        \arrow[]{d}
      \\
      \bullet
        \arrow[]{r}
      &
      \bullet
    \end{tikzcd}\\ \hline\end{tabular}
  \end{center}
\end{example}

\begin{definition}
A \emph{cone} for a diagram $D:\mathbb{G} \to \mathcal{C}$ consists of:
\begin{itemize}
\item an object $X \in \mathcal{C}$
\item together with a family 
$\chi = \setof{\chi_n: X \to Dn}_{n \in \mathbb{G}}$ such that for all 
$m \xrightarrow{e} n$ in $\mathbb{G}$ the diagram 
\begin{center}
  \begin{tikzcd}
    D m
      \arrow["D e"]{rr}
    &
    &
    D n
    \\&&\\
    &
    X
      \arrow["\chi_m"]{uul}
      \arrow["\chi_n",swap]{uur}
    &
  \end{tikzcd}
\end{center}
commutes
\end{itemize}

A \emph{limit} of $D: \mathbb{G} \to \mathcal{C}$ is a terminal cone; that is,
a cone 
\[
\big(L, \lambda = \big\{ \lambda_m: L \to D m \big\}_{m \in \mathbb{G}}\big)
\]
such that for all cones
$\big(X, \chi = \big\{ \chi_m: X \to D m \big\}_{m \in \mathbb{G}}\big)$,
there is a unique $u$ making the following diagram commute:
\begin{center}
  \begin{tikzcd}
    X
      \arrow["\exists!u"]{rr}
      \arrow["\chi_m"below]{rdd}
    &
    &
    L
      \arrow["\lambda_m"]{ldd}
    \\&&\\
    &
    D m
    &
  \end{tikzcd}
\end{center}
\end{definition}

\begin{proposition}
A category with small products and equalizers has all small limits.
\end{proposition}
\begin{proof}[Proof idea]
For a diagram $D$ of shape $\mathbb{G}$, consider the construction
\begin{center}
  \begin{tikzcd}
    L
      \arrow["\text{eq}"]{r}
    &
    \prod_{k\in\mathbb{G}}{D k}
      \arrow
        [
          "{\langle D(e) \comp \pi_m\rangle_{e:m\to n}}"above,
          shift left=+1
        ]
        {rrr}
      \arrow
        [
          "{\langle \pi_n \rangle_{e:m\to n}}"below,
          shift left=-1
        ]
        {rrr}
    &
    &
    &
    \prod_{(e:m\to n) \in \mathbb{G}}
    D n
  \end{tikzcd}
\end{center}

The limit is then given by the family
\begin{center}$ 
\lambda =
\big\{
  \begin{tikzcd}
    L
      \arrow["{\text{eq}}",swap]{r}
      \arrow["\lambda_i"above , bend left=20]{rr}
    &
    {\prod_{k\in\mathbb G} D k}
      \arrow["{\pi_i}",swap,r]
    &
    D i 
  \end{tikzcd}
\big\}_{i\in\mathbb{G}}
$\end{center}
which is a cone because the diagram 
\begin{center}
  \begin{tikzcd}[row sep=5em]
    L
      \arrow["\text{eq}"]{r}
      \arrow[shift left=-1,"D(e)\comp\lambda_m",bend right]{drrr}
      \arrow[shift left=+1,"{\lambda_n}",swap,bend right]{drrr}
    &
    \prod_{k\in\mathbb{G}}{D k}
      \arrow
        [
          "{\langle D(e) \comp \pi_m\rangle_{e:m\to n}}"above,
          shift left=+1
        ]
        {rrr}
      \arrow
        [
          "{\langle \pi_n \rangle_{e:m\to n}}"below,
          shift left=-1
        ]
        {rrr}
      \arrow[shift left=-1,"D(e)\comp\pi_m"]{drr}
      \arrow[shift left=+1,"{\pi_n}",swap]{drr}
    &
    &
    &
    \prod_{(e:m\to n) \in \mathbb{G}}
    D n
    \arrow["\pi_{(e:m\to n)}"]{dl}
    \\
    & & & 
    Dn
    & 
  \end{tikzcd}
\end{center}
commutes for all $e:m\to n$ in $\mathbb G$.

It remains to show the universal property, which is left as an
\textbf{exercise}.
\end{proof}

\begin{lemma}
If a category has pullbacks and terminal object, then it has finite products.
\end{lemma}
\begin{proof}[Proof idea]
For every pullback diagram
\begin{center}
  \begin{tikzcd}
    P
      \arrow[phantom, "\lrcorner", pos=0]{dr}
      \arrow[]{r}
      \arrow[]{d}
    &
    B
      \arrow[]{d}
    \\
    A
      \arrow[]{r}
    &
    1
  \end{tikzcd}
\end{center}
the span
\begin{center}
  \begin{tikzcd}
    &
    P
      \arrow[""]{dl}\arrow[""]{dr}
    &
    \\
    A
    &
    &
    B
  \end{tikzcd}
\end{center}
is a product.
\end{proof}

\begin{lemma}
If a category has pullbacks and terminal object, then it has equalizers.
\end{lemma}
\begin{proof}[Proof idea]
Every pullback 
\begin{center}
  \begin{tikzcd}
    E
      \arrow[phantom, "\lrcorner", pos=0]{dr}
      \arrow["e"]{r}
      \arrow["e"left]{d}
    &
    A
      \arrow["{\langle id_A, g \rangle}"]{d}
    \\
    A
      \arrow["{\langle id_A, f \rangle}"below]{r}
    &
    A \times B
  \end{tikzcd}
\end{center}
gives the equalizer diagram 
\begin{center}
  \begin{tikzcd}
    E
      \arrow["e"above]{r}
    &
    A
      \arrow["f"above, shift left=+1]{r}
      \arrow["g"below, shift left=-1]{r}
    &
    B
  \end{tikzcd}
\end{center}
\end{proof}

\begin{remark}
In $\catset$, the above construction describes the equalizer
\[
  \bigsetof{\,a\in A\ \mbox{\Large$\mid$}\ f(a) = g(a)\,}
\]
by constructing the isomorphic set
\[
  \bigsetof{\, 
    (a, a')\in A\times A\ 
    \mbox{\Large$\mid$}\
    \big(a, f(a)\big) = \big(a', g(a')\big)
    \,}
  \enspace.
\]
\end{remark}

\section{Inverse images}

\begin{proposition}
In a pullback
\begin{center}
\begin{tikzcd}
P \arrow[r] \arrow[d, "p"'] \arrow[phantom, "\lrcorner", pos=0]{dr}  & 
S \arrow[d, "m",tail] 
\\ A \arrow[r] & B
\end{tikzcd}
\end{center}
where $m: S \to B$ is a monomorphism, necessarily so is $p: P \to A$.
\end{proposition}

\begin{remark}
Intuitively, monomorphisms are injective maps and therefore they are like
predicates: they select a subset of the elements in the range. 

In $\catset$, we have the pullback 
\begin{center}
\begin{tikzcd}
\bigsetof{\, (a,s)\in A\times S \suchthat f(a) = m(s) \,}  
\arrow[r] \arrow[d,tail] \arrow[phantom, "\lrcorner", pos=0]{dr}  & 
S \arrow[d, "m", tail] \\
A \arrow[r, "f"'] & B
\end{tikzcd}
\end{center}
and since 
\[ 
  \bigsetof{\, (a,s)\in A\times S \suchthat f(a) = m(s) \,} 
  \ \iso \
  \bigsetof{\, a\in A \suchthat f(a) \in m[S] \,} 
  \ = \
  f^{-1}\big(m[S]\big)
  \enspace,
\]
where $m[S]\eqdef\bigsetof{m(s)\in B\suchthat s\in S}\subseteq B$ is the image
of $S$ under $m$, we also have the pullbacks 
\begin{center}
\begin{tikzcd}
f^{-1}\big(m[S]\big) 
\arrow[r] \arrow[d, hook] \arrow[phantom, "\lrcorner", pos=0]{dr}  
& S \arrow[d,"m",tail] 
\\
A \arrow[r, "f"'] & B
\end{tikzcd}
\qquad and \qquad
\begin{tikzcd}
f^{-1}\big(m[S]\big) 
\arrow[r] \arrow[d, hook] \arrow[phantom, "\lrcorner", pos=0]{dr}  
& m[S] \arrow[d,hook] 
\\
A \arrow[r, "f"'] & B
\end{tikzcd}
\end{center}

Hence, again intuitively, pullbacks of monomorphisms along a map are like
inverse images.

For instance, note that for a morphism $f: A \to B$ we have the pullback
\begin{center}
\begin{tikzcd}
A \arrow[r, "f"] \arrow[d, "id_A",swap,tail]  
\arrow[phantom, "\lrcorner", pos=0]{dr} 
& B \arrow[d, "id_B",tail] \\
A \arrow[r, "f"'] & B
\end{tikzcd}
\end{center}
which abstractly captures the set-theoretic fact that $f^{-1}(B) = A$ for all
functions $f$.

We also have the set-theoretic fact that, for all functions $f:A\to B$ and
$g:B\to C$, 
\[
  (g \circ f)^{-1}(S) 
  \,=\,
  f^{-1}\big(g^{-1}(S)\big) 
  \enspace\text{for all $S\subseteq C$}
\]
that abstractly corresponds to the half of the Pullback Lemma that asserts
that, if 
\begin{center}
\begin{tikzcd}
f^{*}\big(g^{*}(S)\big) 
\arrow[phantom, "\lrcorner", pos=0]{dr} 
\arrow[r] 
\arrow[d,tail] & 
g^{*}(S) \arrow[d,tail] 
\\
A \arrow[r, "f"'] & B 
\end{tikzcd}
\qquad and \qquad
\begin{tikzcd}
g^{*}(S) \arrow[r] \arrow[d,tail]
\arrow[phantom, "\lrcorner", pos=0]{dr} 
& 
S \arrow[d, tail] 
\\
B \arrow[r,"g"'] & C	
\end{tikzcd}
\end{center}
are pullbacks, then so is
\begin{center}
\begin{tikzcd}
f^{*}\big(g^{*}(S)\big) 
\arrow[phantom, "\lrcorner", pos=0]{dr} 
\arrow[r] 
\arrow[d,tail] & g^{*}(S) \arrow[r] & 
S \arrow[d, tail] 
\\
A \arrow[r, "f"'] & B \arrow[r,"g"'] & C	
\end{tikzcd}
\end{center}
\end{remark}

\section{Subobjects}

\begin{proposition}
A morphism $m: A \to B$ is a monomorphism iff 
\begin{center}
\begin{tikzcd}
A \arrow[r, "id_A"] \arrow[d, "id_A"'] \arrow[phantom, "\lrcorner", pos=0]{dr}
& A \arrow[d,"m"] \\ A \arrow[r, "m",swap] & B
\end{tikzcd}	
\end{center}
is a pullback
\end{proposition}

\begin{exercise}
	Prove the proposition.
\end{exercise}

\begin{definition}[Monomorphism equivalence]
Given two monomorphisms $m_1: S_1 \to A$ and $m_2: S_2 \to A$; that is,
	\begin{center}
		\begin{tikzcd}
S_1 \arrow[rd, "m_1"', tail] &  & S_2 \arrow[ld, "m_2", tail] \\
 & A & 
\end{tikzcd}
	\end{center}
\end{definition}
we say $m_1 \approx m_2$ holds iff there exists a (necessarily unique)
isomorphism $S_1 \stackrel\iso\longrightarrow S_2$ such that the diagram
\begin{center}
\begin{tikzcd}
S_1 \arrow[rd, "m_1"', tail] \arrow[rr, "\cong"] &  & S_2 \arrow[ld, "m_2", tail] \\
 & A & 
\end{tikzcd}
\end{center}
commutes.

\begin{definition}[Suboject]
A \emph{suboject} of an object $A$ is an equivalence class $[m]_{\approx}$ for
$m$ a monomorphism into $A$.  
\end{definition}

\begin{definition}[Subobject functor]
For a category $\mathcal{C}$ with pullbacks we have a 
\emph{subobject functor}
\[
\mathrm{Sub}: \mathcal{C}^\op \to  \catset 
\]
mapping objects
\begin{center}
\begin{tikzcd}[column sep=small]
X \arrow[rr, maps to] &  & 
\mathrm{Sub}(X) 
\eqdef \bigsetof{\, [m]_{\approx} \mid m \text{ is a monomorphism into } X \,}  
\end{tikzcd}
\end{center}
and mapping a morphism $f: Y \to X$ in $\mathcal C$ to the function
$\mathrm{Sub}(f): \mathrm{Sub}(X) \to \mathrm{Sub}(Y)$ given by
\begin{center}
\begin{tikzcd}
\lbrack m \rbrack_{\approx} \arrow[r, maps to] & \lbrack f^*(m)\rbrack_{\approx}
\end{tikzcd}
\end{center}
for a pullback square
\begin{center}
\begin{tikzcd}
f^*(S) \arrow[r] \arrow[d, "f^*(m)"', tail] 
\arrow[phantom, "\lrcorner", pos=0]{dr} & S \arrow[d, "m", tail] \\
Y \arrow[r, "f"'] & X
\end{tikzcd}
\end{center}
\end{definition}
\begin{remark}
Functoriality follows from the pullback lemma.	
\end{remark}

\begin{example}
In $\catset$, two injections into the same set are equivalent iff they have
the same image, and $\mathrm{Sub}$ is naturally isomorphic to the
contravariant powerset functor (with action given by inverse image).
\end{example}

\begin{exercise}
What is a representation 
\begin{center}
\begin{tikzcd}
\mathcal{C}(\phold,\Omega) \arrow[r, "\iso", Rightarrow] &
\mathrm{Sub}(\phold)
\end{tikzcd}
\end{center}	
for $\mathrm{Sub}$?
\end{exercise}
 
