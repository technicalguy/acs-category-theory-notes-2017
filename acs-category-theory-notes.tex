%%%%%%%%%%%%%%%%%%%%%%%% NOTES TO CONTRIBUTORS %%%%%%%%%%%%%%%%%%%%%%%%%%%%%%%%%
% - Try to stick to a minimal package set unless you need them. In other       %
%   words, don't use packages you don't understand simply because you're used  %
%   to using them.                                                             %
%                                                                              %
% - $$ eqn $$ is less universal than \[ eqn \] or \begin{equation} eqn         %
%   \end{eqn}. Don't use the math environment just to center things. That's    %
%   what the center environment is for.                                        %
%                                                                              %
% - If you have other stylistic guidelines you want other people to keep in    %
%   mind, put them here.                                                       %
%%%%%%%%%%%%%%%%%%%%%%%%%%%%%%%%%%%%%%%%%%%%%%%%%%%%%%%%%%%%%%%%%%%%%%%%%%%%%%%%

\documentclass[a4paper, 11pt, oneside]{book}

\usepackage{a4wide}
\usepackage{amsthm}
\usepackage{color}
\usepackage{enumitem}
\usepackage{framed}
\usepackage{mathtools}
\usepackage[charter, cal=cmcal]{mathdesign} % displays nicely on computer monitors
\usepackage{microtype}
\usepackage{multicol}
\usepackage{nth}
\usepackage{tikz-cd}
\usepackage{tikz}
\usetikzlibrary{fit, positioning, shapes, calc}
\usepackage[utf8]{inputenc}

\usepackage[T1]{fontenc}
\usepackage[colorlinks=true, linkcolor=blue]{hyperref}
\usepackage{xfrac}

\usepackage{hyperref}
\usepackage{nicefrac}

% Don't put formatting in the Title; this will show up in other places once
% we add headers. Instead, do the formatting manually

\title{ACS L108 Lecture Notes: Category Theory, Type Theory, and Logic}
\author{Marcelo Fiore}
\date{Michaelmas Term 2017--18}

%%%% Environment for Labeling Chapters
\newcommand{\lecturedetails}[2]{
    \vspace*{-10mm}\hspace*{7.75mm}
    \parbox{0.7\textwidth} {
        #2\\
        \textit{#1}
    }
    \vspace{10mm}
}

% make sure that chapters are called lectures
\makeatletter
\renewcommand{\@chapapp}{Lecture}
\makeatother

%%%%%

\theoremstyle{definition}
\newtheorem{definition}{Definition}
\newtheorem{lemma}{Lemma}
\newtheorem{theorem}{Theorem}
\newtheorem{proposition}{Proposition}
\newtheorem{exercise}{Exercise}

\newtheorem*{remark}{Remark}
\newtheorem*{example}{Example}

%%%% Stylistic
\setlength\parindent{0pt}
\setlength\parskip{0.7em}

%%%% Category Theory Macros
\newcommand {\cat}{%
    \mathbf%
}
\newcommand {\domain}[1] {%
    \mathrm{dom}(#1)%
}
\newcommand {\codomain}[1] {%
    \mathrm{cod}(#1)%
}
\newcommand {\idarrow}[1][] {%
    \mathbf{1}{#1}%
}
%%%% Macros for specific categories
\newcommand{\catcat}{\cat {Cat}}
\newcommand{\catmon}{\cat {Mon}}
\newcommand{\catposet}{\cat {Poset}}
\newcommand{\catrel}{\cat {Rel}}
\newcommand{\catset}{\cat {Set}}
\newcommand{\catpfn}{\cat {Pfn}}
\newcommand{\catfinset}{\cat {FinSet}}
\newcommand{\catbij}{\cat {Bij}}
\newcommand{\catgroup}{\cat {Groups}}
\newcommand{\catgraph}{\cat {Graphs}}
\newcommand{\catpreo}{\cat {PreO}}
\newcommand{\catpred}{\cat {Pred}}
\newcommand{\catdirgraph}{\cat {DirGraph}}

%%%% Macros for specific functors
\newcommand{\fnclist}{\mathit{List}}
\newcommand{\fncid}[1]{\mathrm{Id}_{#1}}

%%%% Objects and morphism sets
\DeclareMathOperator{\Ob}{\mathit{Obj}}
\DeclareMathOperator{\Arr}{\mathit{Arr}}

\newcommand{\ie}{\emph{i.e.}}
\newcommand{\etc}{\emph{etc.}}
\newcommand{\viz}{\emph{viz.}}

\newcommand{\eqdef}{\stackrel{\text{def}}{=}}
\newcommand{\iffdef}{\stackrel{\text{def}}{\iff}}
\newcommand{\comp}{\circ} % composition
\newcommand{\icomp}{\,} % implicit composition
\newcommand{\iso}{\cong}
\newcommand{\natiso}{\cong_{nat}}
\newcommand{\adjoint}{\dashv}

\newcommand{\setof}[1]{ \{ #1 \} }
\newcommand{\bigsetof}[1]{ \big\{ #1 \big\} }
\newcommand{\suchthat}{\mid}
\newcommand{\union}{\cup}
\newcommand{\lub}{\bigsqcup} % least upper bound

\newcommand{\nelem}[1]{ \mathbb{ #1 } }
\newcommand{\id}[1]{ \mathrm{id}_{ #1 } }
\newcommand{\Id}[1]{ \mathrm{Id}_{ #1 } }
\newcommand{\idfunc}{ \mathrm{Id} }
\newcommand{\map}[1]{ \mathrm{map}\, #1}
\newcommand{\nats}{\mathbb{N}}
\newcommand{\morpharrow}{\longrightarrow}
\newcommand{\morpharr}[1]{\overset{#1}{\morpharrow}}
\newcommand{\natto}{\Rightarrow}
\newcommand{\xto}[1]{\xrightarrow{#1}}
\newcommand{\xnattto}[1]{\xRightarrow{#1}}
\newcommand{\natarrow}{\Longrightarrow}
\newcommand{\natarr}[1]{\overset{#1}{\natarrow}}
\newcommand{\lscat}[1]{\mathcal{#1}} % locally small category
\newcommand{\scat}[1]{\mathbb{#1}} % small category
\newcommand{\univpair}[1]{\langle #1 \rangle}
\newcommand{\bigunivpair}[1]{\big\langle #1 \big\rangle}
\newcommand{\slice}[2]{\nicefrac{#1}{#2}} % slice category
\newcommand{\powerset}{\mathcal{P}}

\newcommand{\seq}[1]{\langle #1 \rangle}

\newcommand{\yon}{\mathrm{y}}

\newcommand{\verteq}{\rotatebox{90}{$\,=$}}
\newcommand{\equalto}[2]{\underset{\displaystyle\overset{\mkern4mu\verteq}{#2}}{#1}}
\newcommand{\mapsfrom}{\reflectbox{$\;\mapsto\;$}}

\newcommand{\phold}{\text{--}}
\newcommand{\Phold}{\text{=}}

\newcommand{\op}{\mathrm{op}}
\newcommand{\dom}{\mathrm{dom}}

\newcommand{\adjointDown}{\mathbin{\rotatebox[origin=c]{90}{$\vdash$}}}
\newcommand{\adjointUp}{\mathbin{\rotatebox[origin=c]{90}{$\dashv$}}}

% Double-lined fractions, notation for bijective correspondence
% https://tex.stackexchange.com/questions/56003/fraction-with-doubled-line
\newcommand{\Efrac}[2]{%
  \mathchoice
    {\ooalign{%
      $\genfrac{}{}{2.0pt}0{#1}{#2}$\cr%
      $\color{white}\genfrac{}{}{1.0pt}0{\phantom{#1}}{\phantom{#2}}$}}%
    {\ooalign{%
      $\genfrac{}{}{2.0pt}1{#1}{#2}$\cr%
      $\color{white}\genfrac{}{}{1.0pt}1{\phantom{#1}}{\phantom{#2}}$}}%
    {\ooalign{%
      $\genfrac{}{}{2.0pt}2{#1}{#2}$\cr%
      $\color{white}\genfrac{}{}{1.0pt}2{\phantom{#1}}{\phantom{#2}}$}}%
    {\ooalign{%
      $\genfrac{}{}{2.0pt}3{#1}{#2}$\cr%
      $\color{white}\genfrac{}{}{1.0pt}3{\phantom{#1}}{\phantom{#2}}$}}%
}

\newcommand{\efrac}[2]{%
  \mathchoice
    {\ooalign{%
      $\genfrac{}{}{2.0pt}0{\hphantom{#1}}{\hphantom{#2}}$\cr%
      $\color{white}\genfrac{}{}{1.0pt}0{\color{black}#1}{\color{black}#2}$}}%
    {\ooalign{%
      $\genfrac{}{}{2.0pt}1{\hphantom{#1}}{\hphantom{#2}}$\cr%
      $\color{white}\genfrac{}{}{1.0pt}1{\color{black}#1}{\color{black}#2}$}}%
    {\ooalign{%
      $\genfrac{}{}{2.0pt}2{\hphantom{#1}}{\hphantom{#2}}$\cr%
      $\color{white}\genfrac{}{}{1.0pt}2{\color{black}#1}{\color{black}#2}$}}%
    {\ooalign{%
      $\genfrac{}{}{2.0pt}3{\hphantom{#1}}{\hphantom{#2}}$\cr%
      $\color{white}\genfrac{}{}{1.0pt}3{\color{black}#1}{\color{black}#2}$}}%
}

\begin{document}

\begin{center} {\LARGE \sc
Category Theory, Type Theory, and Logic\\
  Lecture Notes\\[4mm]}
  \Large Marcelo Fiore
\end{center}


\thispagestyle{plain}
\begin{multicols}{3}[\section*{Contributors}]
Nathanael Alcock\\
Sebastian Borgeaud\\
Pascal Bose\\
Vishal Chakraborty\\
James Clarke\\
Patrick Fernandes\\
Brad Hardy\\
Andrej Ivašković\\
Razvan Kusztos\\
Stella Lau\\
Nándor Licker\\
Dhruv Makwana\\
Thomas Parks\\
Oliver Richardson\\
Shaun Steenkamp\\
Dima Szamozvancev\\
Zongzhe Yuan\\
\end{multicols}
\clearpage


\tableofcontents

\newpage

\include{tex/lecture1}
\include{tex/lecture2}
\include{tex/lecture3}
\include{tex/lecture4}
\include{tex/lecture5}
\include{tex/lecture6}
\include{tex/lecture7}
\include{tex/lecture8}
\include{tex/lecture9}
\include{tex/lecture10}
\include{tex/lecture11}
\include{tex/lecture12}
\chapter{Adjoint Functors}
\lecturedetails{16 November 2017}{M Fiore, R Kusztos}

\section{More about subobjects classifiers}

Recall the definition of the \underline{Sub} functor: 

\begin{align*}
  \underline{Sub}: \lscat{C}^{op} &\longrightarrow \textbf{Set} \\
  X &\longmapsto \underline{Sub}(X) = \bigl\{ [m]_{\approx} | 
    \textrm{m is mono into X} \bigl\} \\ 
\end{align*}

\begin{center}
  \begin{tikzcd}
    f^{*}(P)
      \arrow[phantom, "\lrcorner", pos=0]{dr} 
      \arrow[r] 
      \arrow[d] & 
    X 
      \arrow[d] \\
    P
      \arrow[r] & 
    Y
  \end{tikzcd}
\end{center}

What is a representation for this?
\begin{definition}[Subobject classifiers]
  To give a natural isomorphism, 
  $\lscat{C}(\_,\Omega) \overset{\iso}{\Longrightarrow} \underline{Sub}(\_) $ 
  is equivalent to giving an 
  $\Omega \in \lscat{C}$ 
  and a 
  $ \Bigl[t \underset{\Omega}{\overset{T}{\downarrow}} \Bigl] 
    \in \underline{Sub}(\Omega)$
  such that for all 
  $X \in \lscat{C}^{\Omega}$
  there exists a bijective correspondence: 

  \begin{align*}
    \forall 
    \Bigl[m \underset{X}{\overset{P}{\downarrow}} \Bigl] 
    \in \underline{Sub}(X) \quad
    \exists! 
    X \overset{\chi_{[m]}}{\longrightarrow} \Omega \quad 
    \textrm{such that} \quad
    \chi_{[m]}^{*}[t] = [m]
  \end{align*}
\end{definition}

\begin{definition}[Subobject classifiers]
  \label{subobject_classifier_def}
  The previous definition is equivalent to simply giving 
  $\Omega \in \lscat{C}$
  and 
  $t \underset{\Omega}{\overset{T}{\downarrow}} \in \lscat{C}$
  such that 
  \begin{align*}
    \forall 
    m \underset{X}{\overset{P}{\downarrow}}
    \in \underline{Sub}(X) \quad
    \exists! 
    X \overset{\chi_{m}}{\longrightarrow} \Omega
  \end{align*}  
  \begin{center}
    \begin{tikzcd}
    P 
      \arrow[phantom, "\lrcorner", pos=0]{dr} 
      \arrow[r, "u"] 
      \arrow[d, "m"] & 
    T 
      \arrow[d, "t"] \\
    X
      \arrow[r, "\chi_m"] & 
    \Omega
    \end{tikzcd}
    \end{center}
  for some $u$
\end{definition}

\begin{example}[in $\textbf{Set}$]
\begin{align*}
  1 \overset{t}{\longrightarrow} \{t, f\} = \Omega
\end{align*}
\begin{align*}
  Sub(X) = \mathcal{P}(X)
\end{align*}

\begin{center}
  \begin{tikzcd}
    P 
      \arrow[phantom, "\lrcorner", pos=0]{dr} 
      \arrow[r, "u"] 
      \arrow[d, "m"] & 
    1
      \arrow[d, "t"] \\
    X
      \arrow[r, "\exists!\varphi"] & 
    \{t, f\}
    \end{tikzcd}
  \end{center}

Where $\varphi$ is such that $\varphi^{-1}\{t\} = P$ \\
Since we are in set, $\varphi$ is the characteristic function of X

\begin{equation*}
 \varphi = \chi_m(x) = 
  \begin{cases}
    t & x \in P \\
    f & x \notin P 
  \end{cases}
\end{equation*}

\end{example}

\begin{exercise}
  In the subobject classifier definition 
  (Definition \ref{subobject_classifier_def}), $T$ is necessarily a
terminal object
\end{exercise}

\begin{definition}[Toposes]
  A topos is a category with: 
  \begin{itemize}
    \item all finite limits 
    \item exponentials
    \item subobject classifier $\Omega$
  \end{itemize}

Examples of toposes are: 
  \textbf{Set}, 
  $\textbf{Set}^{\mathbb{C}^{op}}$, 
  \textbf{DirGraph}

\end{definition}

\begin{remark}
  Toposes are models of Higher Intuitionistic Logic.
\end{remark}

\begin{exercise}
  What is $\Omega$ in \textbf{DirGraph}?

  For some graph 
  $G = (N, E)$, 
  Sub(G) = the set of all its subgraphs, each given by a subset of nodes,
  $S_n \subseteq N$
  and a subset of edges 
  $E_n \subseteq E$ 
  between nodes in $S_n$.

  \begin{center}
    \begin{tikzcd}
      \bullet 
        \arrow[rr, bend left] & {} \arrow[dd, hook'] & 
        \bullet \arrow[loop, looseness=4] &  &  &  & 
        \bullet \arrow[dd, hook'] \arrow[loop, looseness=4] \\
       &  &  &  &  &  &  \\
       & {} &  &  &  &  & {} \\
      \bullet \arrow[rr, bend left] 
      \arrow[rd, bend right] &  & 
      \bullet
        & {} 
      \arrow[rr] &  & {} & 
      \textcolor{green}{\bullet} 
      \arrow[loop left, green]
      \arrow[loop right, red]
      \arrow[bend left=20, blue]{d} \\
       & \bullet \arrow[ru, bend right] \arrow[lu, bend right] &  &  &  &  & 
       \textcolor{red}{\bullet} 
       \arrow[loop right, red]
       \arrow[bend left=20, blue]{u}
    \end{tikzcd}
  \end{center}
\end{exercise}

\begin{exercise}
  What is $\Omega$ in $\textbf{Set}^{(0 \to 1)}$, also known as the 
Sierpinski topos.
\end{exercise}

\section{Adjoint Functors}

\begin{example}
  In the first lecture we introduced the free functor which maps a set to 
the free monoid:

  \begin{align*}
    \textbf{Set} &\overset{F}{\longrightarrow} \textbf{Mon} \\
    S &\longmapsto (S^*, t, \circ)
  \end{align*}
  Similarly, we can introduce the forgetful functor U, which gives the set 
underlying a monoid.

\begin{center}
  \begin{tikzcd}[sep=3em]
    \textbf{Mon}
      \ar["U"{right}]{d} 
    \\
    \textbf{Set}
      \ar[bend left=40, "F"{left}]{u}
  \end{tikzcd}\\[1mm]
\end{center}

Here, we observe that:

\begin{align*}
  \textbf{Mon}(F(S), \underline{M}) &\cong_{nat} 
    \textbf{Set}(S, U(\underline(M))) \\
    \shortintertext{We write that:} \\
    F &\dashv U \\
    \intertext{To say that F and U are adjoints; F is called the left adjoint
    whereas U is called the right adjoint}
  \end{align*}
\end{example}

\begin{example}[Colimits]
  \begin{align*}
    \lscat{C} &\underset{\Delta}{\longrightarrow} \lscat{C}^\mathbb{G} \\ 
    X &\longmapsto \Delta(X): \mathbb{C} \to \lscat{G} \quad 
    \textrm{which maps} \quad
    e \overset{m}{\underset{n}{\downarrow}} \longmapsto
    \overset{x}{\underset{x}{\downarrow}}id
  \end{align*} 
That is, 
  \begin{align*}
    \textrm{For} \quad D \in \lscat{C}^\mathbb{G},  \quad
    &\varphi: D \Rightarrow \Delta(X) \\
    &\equiv \textrm{cocone with vertex x for diagram D} \\
    \lscat{C}(\underline{colim}D, X) &\cong 
      \lscat{C}^{\mathbb{G}}(D, \Delta(X)) \\ 
    \underline{colim} &\dashv \Delta \\ 
    \shortintertext{Dually,} \\ 
    \lscat{C}(X, \underline{lim}D) &\cong
      \lscat{C}^{\mathbb{G}}(\Delta(X), D) \\
    \Delta &\dashv \underline{lim}  
  \end{align*}

\end{example}

\begin{definition}[Adjoint Pair]
An adojoint pair of functors, $F \dashv G: \lscat{D} \to \lscat{C}$ consists
of: 
 \begin{center}
  \begin{tikzcd}
    \lscat{D} 
    \ar[bend left=20, "G"]{r} &
   \lscat{C} 
    \ar[bend left=20, "F"]{l}
  \end{tikzcd}
\end{center}
together with a natural isomorphism:
\begin{align*}
\lscat{D}(F C, D) &\cong 
  \lscat{C}(C, G D) \\ 
\end{align*}

Equivalently, to give a left adjoint $F$ to a functor 
$G: \lscat{D} \to \lscat{G}$
is to give for all $X \in \lscat{C}$
and object $F(x) \in \lscat{D}$
together with a map $X \underset{\eta_X}{\rightarrow} G(F(X)$
in $\lscat{C} $
such that: 
\begin{center}
    \begin{tikzcd}[ampersand replacement=\&]
    X \arrow[r, "\eta_X"]
    \arrow[dr, "\forall f"{left}] \&
    G(F(X)) \arrow[d, "G(f^\#)"] \& 
    F(X) \arrow[d, mapsto, "\exists! f^\#"] \\
    {} \& G(Y) \& Y
  \end{tikzcd} 
\end{center}
\end{definition}
\begin{proposition}
  Left adjoints preserve colimits and dually, right adjoints preserve limits.
\end{proposition}

\begin{example}
A functor $F: \lscat{C} \to \lscat{D}$ preserves sums if:

  \begin{center}
    \begin{tabular}[t]{lc}
    $\forall$ sums in $\lscat{C}$
    \begin{tikzcd}
      {} & A + B & {} \\
      A \ar["\iota_1"]{ru} & & B \ar["\iota_2"above]{lu} \\
    \end{tikzcd}
    ,
    \begin{tikzcd}
      {} & F(A + B) & {} \\
      F(A) \ar["F(\iota_1)"]{ru} & & F(B) \ar["F(\iota_2)"above]{lu} \\
    \end{tikzcd}
    is a sum in $\lscat{D}$.
    \end{tabular}
  \end{center}

If $\lscat{D}$ has sums, 

\begin{center}
  \begin{tikzcd}
    & F(A) + F(B) \ar["\cong", dotted]{d} & \\ 
    & F(A + B) & \\
    F(A) \ar["F(\iota_1)"]{ru} 
         \ar["\iota^{F(A), F(B)}_1", bend left=40]{ruu} & & 
    F(B) \ar["F(\iota_2)"]{lu}
         \ar["\iota^{F(A), F(B)}_2", bend right=40]{luu} \\
    

  \end{tikzcd}
\end{center} 

We need to show that $F(A) + F(B) \cong F(A + B)$. We have previously shown
that this is equivalent to showing that: 

\[ 
  \Efrac
    {F(A) + F(B) \to z}
    {F(A + B) \to z}
\]

Intuitively,

\LARGE\[
  \Efrac
  {
  \Efrac
    {F(A) + F(B) \to z} 
    { \Efrac 
      {F(A) \to z}  
      {A \to G z}
      \quad 
      \Efrac 
      {F(B) \to z}
      {B \to G z}
    }
  }
  {
    \Efrac 
    {A + B \to G z}
    {F(A + B) \to z}
  }
\]

\end{example}

\begin{example}
  Exponentials are adjoints.
  \begin{align*}
    \_ \times A &\dashv A \Rightarrow (\_) \\
    \lscat{C}(X (\times A), B) &\cong \lscat{C}(X, (A \Rightarrow) B)
  \end{align*}
\end{example} 

\chapter{Adjoint Functors in Logic}
\lecturedetails{21 November 2017}{M Fiore, B Hardy, O Richardson}

The agenda for the next couple of lectures is (1) to focus on the links to logic (which will be presented in an informal manner, and not in full generality), and (2) some examples and techniques for how to prove things with adjoint functors.

\section{Different Definitions of Adjunctions}
We will explore four equivalent definitions of adjoint functors; the first three will be given in this lecture, and the last one will be described later on, once we have more intuition and context.

\subsection{Definition By Isomorphism}
The first definition is the original one given in in the last lecture. Given two functors $F: \lscat{C} \to \lscat{D}$ and $U: \lscat{D} \to \lscat{C}$ (think of $F$ as the free functor, which creates the most general possible structure, and $U$ as forgetful) 
 \begin{center}
  \begin{tikzcd}
    \lscat{D} 
    \ar[bend left=20, "G"]{r} &
   \lscat{C} 
    \ar[bend left=20, "F"]{l}
  \end{tikzcd}
\end{center}
Recall that $U \vdash F$ if there is an isomorphism:
\begin{equation*}
\lscat{D}(F C, D) \cong 
  \lscat{C}(C, G D) \;.
\end{equation*}
which is natural in $C$ and $D$. In the isomorphism above, let's name the maps $\varphi$ and $\phi$, as in the diagram below.
\begin{center}
  \begin{tikzcd}
    \lscat{D}(FC, D)
    \ar[bend left=20, "\varphi"]{r} &
    \lscat{C}(C, UD)
    \ar[bend left=20,  "\phi"]{l}
  \end{tikzcd}
\end{center}
We can also view these equivalences as two bijective correspondances between morphisms:
\begin{align*}
\Efrac{ F C \overset{f}\longrightarrow D}
{C \underset{\varphi(f)}\longrightarrow UD}
&\hspace{2in}
\Efrac{F C \overset{f}\longrightarrow D \overset{h}\longrightarrow D'}
{C \underset{\varphi(f)}\longrightarrow UD \underset{\varphi(U h)}\longrightarrow UD'}
\end{align*}
The question now becomes: what does naturality in $C$ look like? For any morphism $h : C' \to C$ in $\lscat{C}$, naturality in $C$ amounts to a structure preserving commutativity condition on $\varphi$, as below
\begin{align*}
  \Efrac{F C' \overset{Fh}\longrightarrow F C \overset{f}\longrightarrow D}
  {C' \underset{h}\longrightarrow C \underset{\varphi(f)}\longrightarrow UD}
  &\hspace{2in}
  \Efrac{F C \overset{f}\longrightarrow D \overset{h}\longrightarrow D'}
    {C \underset{\varphi(f)}\longrightarrow UD \underset{\varphi(U h)}\longrightarrow UD'}
  \\[1em]
  \text{Naturality in $C$:}~~&\hspace{2in}~~\text{Naturality in $D$:} \\
  \varphi(f \comp F h) = \varphi(f) \comp h
  &\hspace{2in}
    \varphi(h \comp f) = Uh \comp \varphi(f)
\end{align*}

\begin{remark}
	To ``transpose'' an arrow is to move it from the top to the bottom of the equivalence (or vice versa). On the left, this amounts to applying $F$, and on the right it ammounts to applying $U$.
\end{remark}

\subsection{Definition by Units}

In particular, we can transpose identities, and so we have $\varphi(\id{FC}) = \eta_C$ satisfying, for any $f$,
\begin{equation*}
  \Efrac{FC' \overset{F f}\longrightarrow FC \overset{\id{}}\longrightarrow FC}
  {C' \underset{f}\longrightarrow C \underset{\eta_C}\longrightarrow UFC}\;.
\end{equation*} 



\begin{definition}[Adjunctions, version 2]
Putting that last fact differently, we have an equivalent definition of an
adjunction: to have an adunction $F \dashv U$ is to have for each $C$, $\eta_C$ such that
\begin{equation*}
  \Efrac{FC \overset{\id{}}\longrightarrow FC}
  {C \underset{\eta_C}\longrightarrow UFC}
\end{equation*} 
and
\begin{center}
\begin{tikzcd}
C \arrow[rdd, "\forall f"'] \arrow[rr, "\eta_C"] &  & UFC \arrow[ldd, "Uf^\#"] & UC \arrow[dd, "\exists!f^\#"] \\
 &  &  &  \\
 & UD &  & D
\end{tikzcd}
\end{center}
\end{definition}

Thus, an an adjunction can be specified by a family of identity transpose maps $\eta_C$.

\subsection{Definition by Counits}
\begin{definition}[Adjunctions, version 3]
Similarly, we can play the dual game to arrive at another equivalent defintion:
to have an adunction $F \dashv U$ is to have for each $D$, $\epsilon_D$ such that
\begin{equation*}
  \Efrac{FUD \overset{\epsilon_D}\longrightarrow D}
  {UD \underset{\id{}}\longrightarrow UD}
\end{equation*} 
and
\begin{center}
\begin{tikzcd}
UD & FUD \arrow[rr, "\varepsilon_D"] &  & D \\
 &  &  &  \\
C \arrow[uu, "\exists! \hat f"] &  & FC \arrow[luu, "F \hat f"] \arrow[ruu, "\forall f"'] & 
\end{tikzcd}
\end{center}
\end{definition}

\section{Adjoints Between Posets}
\begin{remark}
	Sometimes if struggling with understanding a category in full generality, one can take away structure, and think of it as an order or a monoid. 
\end{remark}
If we regard posets as categories, we can talk about adjunctions between them. note that adjuncts between posets are also called ``Galois connections'', and have applications in abstract interpretations.


Suppose we have
\begin{center}
\begin{tikzcd}
P \arrow[rr, "f", bend left] & \adjointDown & Q \arrow[ll, "g", bend left]
\end{tikzcd}
\end{center}
(recalling that functors $f,g$ between poset categories are monotone functions
between the posets).

Then to have the adjunction is to have
\begin{itemize}
\item From the bijection condition (definition 1)
\[\forall p \in P, q \in Q.\; f(p) \le_Q q \iff p \le_p g(q)\;;\]
\item by naturality in $p$, (or more clearly, units from definition 2)
\[\forall p \in P.\; p \le_P (f \comp g)(p)\;;\]
\item by naturality in $q$, (or more clearly, counits from definition 3)
\[\forall q \in Q.\; (f \comp g)(q) \le_Q q\;.\]
\end{itemize}

Now, suppose $P$ and $Q$ are complete (\ie they have all joins and meets).
Notice that $f$ preserves joins and $g$ preserves meets. Conversely (this only
works in $\catposet$), if $f : P \longrightarrow Q$ preserves joins then it has
a right adjoint, and if $g : Q \longrightarrow P$ preserves meets then it has a
left adjoint.

\emph{Proof}. Given $g : Q \longrightarrow P$, we need to define $f : P
\longrightarrow Q$ such that
\begin{center}
  \begin{tikzcd}
p \arrow[r, "f"] \arrow[d] & f(p) \arrow[d] \\
g(q) & q \arrow[l, "g"]
\end{tikzcd}
\end{center}
commutes.

\begin{exercise}
Check that the greatest lower bound $f$ and least upper bound $g$ 
\begin{align*}
	f &\eqdef \bigwedge \{q \in Q~|~ p \le g(q)\}\\
	g &\eqdef \bigvee \{p \in P~|~ f(p) \le q\}
\end{align*}
are adjoint.
\end{exercise}

\section{Adjoints as a basis for Existential/Universal Quantifies}
\begin{example}
  Let $X \overset{f}\longrightarrow Y$ be a function (in $\catset$). Because the
  inverse image $f^{-1}$ preserves both $\cup$ and $\cap$ (meets and joins), then by the above it must have both a right and a left adjoint functor. Hence we have arrows
  which we will call $\exists f$ and $\forall f$ satisfying
  \begin{center}
    \begin{tikzcd}
 &  & \adjointDown &  &  \\
\mathcal P(X) \arrow[rrrr, "\exists f", bend left=70] \arrow[rrrr, "\forall f"', bend right=70] &  &  &  & \mathcal P(Y) \arrow[llll, "f^{-1}"] \\
 &  & \adjointDown &  & 
\end{tikzcd}
  \end{center}

  \begin{exercise}
    Give explicit descriptions of $\exists f \dashv f^{-1} \dashv \forall f$.
  \end{exercise}

  \textbf{Example of the example.} Let's compute the adjoints in the special case of the above example, where $X = A \times B$, $Y = B$, and $f = \pi_2^{A,B}$.
  
  \begin{center}
    \begin{tikzcd}
 &  & \adjointDown &  &  \\
\mathcal P(A \times B) \arrow[rrrr, "\exists_A", bend left=70] \arrow[rrrr, "\forall_A"', bend right=70] &  &  &  & \mathcal P(B) \arrow[llll, "\pi_2^{-1}"] \\
 &  & \adjointDown &  & 
\end{tikzcd}
  \end{center}

\textbf{Idea.} Members of $\mathcal P(A \times B)$ can be seen as predicates/relations
over $A$ and $B$. Then, given some predicate $R(x^A, y^B)$ (read as $x$ fo type $A$, and $y$ of type $B$), $\exists_A$ and
$\forall_A$ give us predicates on just a variable in $B$:
\begin{align*}
  \exists_A(P)(y^B) & = \exists x \in A.~R(x, y) \\
  \forall_A(P)(y^B) & = \forall x \in A.~R(x, y).
\end{align*}

Notice that $\pi_2^{-1}(P) = \{(a, b) \in A \times B \;|\; b \in P\} \longrightarrow
P = A \times P$. Then to show that $\exists_A$ and $\forall_A$ are the required
adjoints is to show that

\[
  \Efrac{\exists_A(R) \subseteq P}{R \subseteq \pi_2^{-1}(P) = A \times P}
\]
and
\[
  \Efrac{A \times P = \pi_2^{-1}(P) \subseteq R}{P \subseteq \forall_A(R)}\;.
\]

Interestingly, this is a possible definition of quantifiers in terms of
adjoints [due to Lawvere].
\end{example}

\begin{remark}
 This works in every topos with \underline{Sub} (a subobject classifier), although the details are different.
\end{remark}

\section{Connectives and Slice Categories}

Observe the following two analogies

\begin{align*}
	\Efrac{C \to A \times B}{C \to A\quad C \to B} &\hspace{10em} \Efrac{\Gamma \vdash A \land B}{\Gamma \vdash A\quad \Gamma \vdash B} \\
	\Efrac{C \times A  \to B}{C \to A \Rightarrow B} &\hspace{10em} \Efrac{\Gamma, A \vdash B}{\Gamma \vdash A \Rightarrow B}
\end{align*}

\begin{remark}
	This generalizes in connection to type theory, with respect to slice categories (to be defined below), where $\exists_A$ becomes a $\Sigma$ type, and $\forall_A$ becomes a $\Pi$ type.
\end{remark}

\begin{definition}
	Given a category $\lscat{C}$ and an object $A \in \lscat C$, the \textit{slice of $\lscat C$ over $A$}, denoted $\sfrac{\lscat C}{ A}$, has
	\begin{align*}
		\textbf{objects:} \qquad & \left(X \in \lscat C, ~f\underset{A}{\overset{X}{\downarrow}} \right) \\
		\textbf{Morphisms:} \qquad & \left(X , ~f \underset{A}{\overset{X}{\downarrow}} \right) \overset{h}{\longrightarrow} \left(Y, ~g\underset{A}{\overset{Y}{\downarrow}} \right)\\
		& \text{ are maps $h: X \to Y$ such that } \\
		& \begin{tikzcd}[ampersand replacement=\&]
			 X \arrow[r, "h"] \arrow[dr, "f"'] \& Y \arrow[d, "g"] \\ \&A
		\end{tikzcd} \qquad\text{commutes}
	\end{align*}
\end{definition}

We also have a forgetful functor $\Sigma_A$, defined as follows
\begin{align*}
\sfrac{\lscat C }{A} &\overset{\Sigma_A}{\longrightarrow} \lscat C \\
\left(X , \underset{A}{\overset{X}{\downarrow}}f \right) &\longmapsto X
\end{align*}

\textbf{Proposition.} $\Sigma_A: \sfrac{\lscat C}{A} \to \lscat C$ has a right adjoint if and only if $\lscat C$ has products with A. \footnote{this condition was originally "$\lscat C$ has pullbacks along maps into $A$, but we didn't have time enough to do the more general result with this premise}

\begin{center}
	\begin{tikzcd}[column sep = huge, row sep=large]
		\sfrac{\lscat C}{A}
			\arrow[r, bend left= 50, "\Pi_A"{name=P, above}]
			\arrow[r, bend right=50, "\Sigma_A"{name=S, below}]
		 & \lscat C \arrow[l, "{(~)^*}"{name=M, description}]\\
		 \arrow[from={S}, to={M}, phantom, "\dashv" rotate=90 ]
		 \arrow[from={M}, to={P}, phantom, "\dashv" rotate=90 ]
		 
	\end{tikzcd}
\end{center}


In Type Theory, given a function $f: A \to B$ in $\lscat C$ that has pullbacks $f^*$ along $f$, then the sigma is left adjoint to $f^*$, and if we're lucky we will also have $\pi_f$, right adjoint $f^*$.
\begin{center}
	\begin{tikzcd}[column sep = huge, row sep=large]
		\sfrac{\lscat C}{A}
			\arrow[r, bend left= 50, "\Pi_f"{name=P, above}]
			\arrow[r, bend right=50, "\Sigma_f"{name=S, below}]
		& \sfrac{\lscat C}{B} \arrow[l, "{f^*}"{name=M, description}]\\
		\arrow[from={S}, to={M}, phantom, "\dashv" rotate=90 ]
		\arrow[from={M}, to={P}, phantom, "\dashv" rotate=90 ]
	\end{tikzcd}
\end{center}
The general idea is to replace the subobjects with slices and keep the pullbacks.
\begin{align*}
	\lscat C &\overset{(~)^*}{\longrightarrow} \sfrac{\lscat C }{A} \\
	X &\longmapsto \left(X \times A, \underset{A}{\overset{X \times A}{\downarrow}}\pi_2 \right)
\end{align*}

\chapter{Categorical Semantics of Dependent Sums and Products; Monads}
\lecturedetails{23 November 2017}{M Fiore, N Alcock}

\section{Categorical Semantics of Dependent Sums and Products}
There is a unique map $X \to 1$, so $\slice{\cat{C}}{1}: (X, X \to 1)$ is equivalent to simply giving $X$. Therefore $\slice{\cat{C}}{1} \iso \cat{C}$.

\begin{center}
\begin{tikzcd}
\slice{\cat{C}}{A} \arrow[rr, "(A \Rightarrow -) = \prod_A"{name=U}, bend left=49] \arrow[rr, "\sum_A"'{name=D}, bend right=49] &  & \cat{C} \arrow[ll, "-\times A"{name=M}] &  & A \times C \arrow[d] \arrow[r] \arrow[dr, phantom, "\lrcorner", very near start] & C \arrow[d] \\
 &  &  &  & A \arrow[r] & 1
\arrow[from=U, to=M, phantom, "\dashv" rotate=90]
\arrow[from=M, to=D, phantom, "\dashv" rotate=90]
\end{tikzcd}
\end{center}

\begin{example}
In $\catset$ we have: $I \xto{f} J$.

\begin{center}
\begin{tikzcd}
\slice{\catset}{I} \arrow[rr, "\prod_f"{name=U}, bend left=49] \arrow[rr, "\sum_f"'{name=D}, bend right=49] &  & \slice{\catset}{J} \arrow[ll, "f^{*}"{name=M}]
\arrow[from=U, to=M, phantom, "\dashv" rotate=90]
\arrow[from=M, to=D, phantom, "\dashv" rotate=90]
\end{tikzcd}
\end{center}

\begin{align*}
\{X_i\}_{i \in I} \xmapsto{\sum_f} \{\sum_{i : f(i) \neq j} X_i\}_{j \in J} \\
\{Y_{f(i)}\}_{i \in I} \overset{f^{*}}{\mapsfrom} \{Y_j\}_{j \in J} \\
\{X_i\}_{i \in I} \xmapsto{\prod_f} \{\prod_{i : f(i) \neq j} X_i\}_{j \in J}
\end{align*}

$f^{*}$ is sometimes known as \textit{reindexing}.
\end{example}

Consider  $\slice{\catset}{I}$. Objects are sets $X$ with maps into $I$: $(X, X \xto{p} I)$. $X = \bigcup_{i \in I} \{x \in X \suchthat p(x) = i\}$. To make this precise, we can consider:

\begin{center}
\begin{tikzcd}
\slice{\catset}{I} \arrow[rr, bend left=49] &  & \catset^I \arrow[ll, bend left=49]
\end{tikzcd}
\end{center}

where $\catset^I$ is a category with objects $X = \{X_i\}_{i \in I}$ and morphisms such that for every morphism $S \xto{\varphi} T$ in $\catset$, we have a morphism $S_i \xto{\varphi_i} T_i$ for each $i \in I$.

\begin{center}
\begin{tikzcd}
X \arrow[d, "p"] \arrow[r, maps to] & \{p^{-1}(i)\}_{i \in I} = X_i &  & \bigcup_{i \in I} (\{i\} \times S_i) = \biguplus_{i \in I} S_i \arrow[d] &  & S=\{S_i\}_{i \in I} \arrow[ll, maps to] \\
I &  &  & I &  & 
\end{tikzcd}
\end{center}

\section{Equivalences of categories}

\begin{definition}[Equivalence of categories]
Let $F: \cat{C} \to \cat{D}$ and $G: \cat{D} \to \cat{C}$ be adjoint functors, $F \adjoint G$. If there exist natural transformations such that $\Id{\cat{C}} \iso GF$ and $FG \iso \Id{\cat{D}}$, then $\cat{C}$ and $\cat{D}$ are said to be equivalent.
\end{definition}

$\slice{\catset}{I}$ and $\catset^{I}$ are equivalent.

\begin{center}
\begin{tikzcd}
X \arrow[r, maps to] \arrow[d] & \{p^{-1}(i)\}_{i \in I} \arrow[r, maps to] & \biguplus_{i \in I} p^{-1}(i) \arrow[r, "\cong" description, phantom, no head] \arrow[d] & X \arrow[ld] \\
I &  & I &  \\
\{S_i\}_{i \in I} \arrow[r, maps to] & \biguplus_{i \in I} S_i \arrow[r, maps to] \arrow[d] & \{\{i\} \times S_i\}_{i \in I} \arrow[r, "\cong" description, phantom, no head] & \{S_i\}_{i \in I} \\
 & I &  & 
\end{tikzcd}
\end{center}

Equivalence is the natural notion for comparing categories.

For $\scat{C}, \scat{D}$ small categories:

\begin{center}
\begin{tikzcd}
\catset^{\scat{C}^{op}} \arrow[rr, "f_{*}"{name=U}, bend left=49] \arrow[rr, "f!"'{name=D}, bend right=49] &  & \catset^{\scat{D}^{op}} \arrow[ll, "f^{*}"{name=M}] \\
\scat{C} \arrow[rr, "f"'] \arrow[u, hook, "y"] & & \scat{D} \arrow[u, hook, "y"]
\arrow[from=U, to=M, phantom, "\dashv" rotate=90]
\arrow[from=M, to=D, phantom, "\dashv" rotate=90]
\end{tikzcd}
\end{center}

where $f^{*}$ maps $(\scat{D}^{op} \xto{X} \catset) \mapsto (\scat{C} \xto{Xf} \catset)$, and $y$ is the Yoneda embedding. This is the categorical version of:

\begin{center}
\begin{tikzcd}
\powerset(A) \arrow[rr, ""{name=U}, bend left=49] \arrow[rr, ""{name=D}, bend right=49] &  & \powerset(B) \arrow[ll, ""{name=M}] \\
A \arrow[rr, "f"'] \arrow[u, hook, "\{-\}"] & & B \arrow[u, hook, "\{-\}"']
\arrow[from=U, to=M, phantom, "\dashv" rotate=90]
\arrow[from=M, to=D, phantom, "\dashv" rotate=90]
\end{tikzcd}
\end{center}

\section{Adjoint functors}
To recap, the following three conditions are equivalent:
\begin{align*}
\cat{D}(FX, Y) \natiso \cat{C}(X, GY) \\
X \xto{\eta_X} GFX & \text{(univ.)} \\
FGY \xto{\eta_Y} Y & \text{(univ.)} \\
\end{align*}

For $X \to Y$, if $\eta$ and $\varepsilon$ are natural transformations, then the following diagram commutes:

\begin{center}
\begin{tikzcd}
X \arrow[r, "\eta_X"] \arrow[d, "f"'] & GFX \arrow[d, "GFf"] \\
Y \arrow[r, "\eta_Y"] & GFY
\end{tikzcd}
\end{center}

There is a general proof technique for proving diagrams commute using adjunctions. Say that we want to prove the following diagram commutes:

\begin{center}
\begin{tikzcd}
X \arrow[rd, bend left=49] \arrow[rd, bend right=49] &  \\
 & GZ
\end{tikzcd}
\end{center}

This diagram commutes iff the following one does:

\begin{center}
\begin{tikzcd}
X \arrow[rd, "F", bend left=49] \arrow[rd, "F"', bend right=49] &  &  \\
 & FGZ \arrow[rd, "\varepsilon_Z"] &  \\
 &  & Z
\end{tikzcd}
\end{center}

Similarly, if we want to prove the following diagram commutes:

\begin{center}
\begin{tikzcd}
FA \arrow[rd, bend left=49] \arrow[rd, bend right=49] &  \\
 & B
\end{tikzcd}
\end{center}

we can use the commutativity of the following one:

\begin{center}
\begin{tikzcd}
A \arrow[rd, "\eta_A"] &  &  \\
 & GFA \arrow[rd, "G", bend left=49] \arrow[rd, "G"', bend right=49] &  \\
 &  & GB
\end{tikzcd}
\end{center}

\begin{example}
$ $ \newline
\begin{center}
\begin{tikzcd}
FX \arrow[r, "F\eta_X"] \arrow[d, "Ff"] & FGFX \arrow[d, "FGFf"] &  \\
FY \arrow[r, "F\eta_Y"] & FGFY \arrow[rd, "\varepsilon_Y"] &  \\
 &  & FY
\end{tikzcd}
\end{center}

\[\Efrac{FX' \xto{Fh} FX \xto{f} Y}{X' \xto{h} X \xto{\hat{f}} G}\]
\[(f \comp Fh)\hat{} = \hat{f} \comp h\]
So:
\[\Efrac{FX \xto{\id{FX}} FX}{X \xto{\eta_X \iso \hat{\id{FX}}} GFX}\]

\begin{definition}[Adjunction triangle laws]
For $F \adjoint G$, the following diagrams commute (making use of the proof technique above):

\begin{center}
\begin{tikzcd}
G \arrow[Rightarrow, rd, "\id{G}"'] \arrow[Rightarrow, r, "\eta G"] & GFG \arrow[Rightarrow, d, "G\varepsilon"] & F \arrow[Rightarrow, rd, "\id{F}"'] \arrow[Rightarrow, r, "F\eta"] & FGF \arrow[Rightarrow, d, "\varepsilon F"] \\
 & G &  & F
\end{tikzcd}
\end{center}

In fact, to have an adjunction is equivalent to having:
\begin{center}
\begin{tikzcd}
\cat{C} \arrow[r, "F", bend left] & \cat{D} \arrow[l, "G", bend left]
\end{tikzcd}
\end{center}
together with $\eta: \Id{} \natto GF$ and $\varepsilon: FG \natto \Id{}$ such that the triangle laws hold.
\end{definition}
\end{example}

\section{Monads}
We want to compare $(FX \xto{f} Y) \mapsto (X \xto{\eta_X} GFX \xto{Gf} GY)$ and $(FX \xto{Fg} FGY \xto{\varepsilon_Y} Y) \mapsfrom (X \xto{g} GY)$ to the identity.

\begin{definition}[Monad]
A monad on $\cat{C}$ is an endofunctor $T$ together with $\eta: \Id{} \natto T$, $\mu: TT \natto T$ satisfying the monoid laws:

\textbf{Identity:}
\begin{center}
\begin{tikzcd}
T \arrow[rd, "\id{}"] \arrow[r, "T\eta"] & TT \arrow[d, "\mu"] & T \arrow[ld, "\id{}"'] \arrow[l, "\eta T"'] \\
 & T & 
\end{tikzcd}
\end{center}

\textbf{Associativity:}
\begin{center}
\begin{tikzcd}
TTT \arrow[r, "T\mu"] \arrow[d, "\mu T"'] & TT \arrow[d, "\mu"] \\
TT \arrow[r, "\mu"] & T
\end{tikzcd}
\end{center}
\end{definition}

If we pick $T = GF$, then the monoid laws are satisfied by the adjunction's triangle laws (for example, $TT \xto{\mu} T$ is equal to $GFGF \xto{G \varepsilon F} GF$).

\begin{proposition}
Every adjunction, $F: \cat{D} \to \cat{C} \adjoint G: \cat{C} \to \cat{D}$ induces a monad on $\cat{C}$, where $T = GF$.

\begin{center}
\begin{tikzcd}
\cat{D} \arrow[r, "F"{name=F}, bend left] & \cat{C} \arrow[l, "G"{name=G}, bend left] \arrow[loop right, "T"]
\arrow[from=F, to=G, phantom, "" description]{}{\perp}
\end{tikzcd}
\end{center}
\end{proposition}

\end{document}
