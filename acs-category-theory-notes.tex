%%%%%%%%%%%%%%%%%%%%%%%% NOTES TO CONTRIBUTORS %%%%%%%%%%%%%%%%%%%%%%%%%%%%%%%%%
% - Try to stick to a minimal package set unless you need them. In other       %
%   words, don't use packages you don't understand simply because you're used  %
%   to using them.                                                             %
%                                                                              %
% - $$ eqn $$ is less universal than \[ eqn \] or \begin{equation} eqn         %
%   \end{eqn}. Don't use the math environment just to center things. That's    %
%   what the center environment is for.                                        %
%                                                                              %
% - If you have other stylistic guidelines you want other people to keep in    %
%   mind, put them here.                                                       %
%%%%%%%%%%%%%%%%%%%%%%%%%%%%%%%%%%%%%%%%%%%%%%%%%%%%%%%%%%%%%%%%%%%%%%%%%%%%%%%%

\documentclass[a4paper, 11pt, oneside]{book}

\usepackage{a4wide}
\usepackage{amsthm}
\usepackage{color}
\usepackage{enumitem}
\usepackage{framed}
\usepackage{mathtools}
\usepackage[charter, cal=cmcal]{mathdesign} % displays nicely on computer monitors
\usepackage{microtype}
\usepackage{multicol}
\usepackage{nth}
\usepackage{tikz-cd}
\usepackage{tikz}
\usetikzlibrary{fit, positioning, shapes, calc}
\usepackage[utf8]{inputenc}
\usepackage{hyperref}

% Don't put formatting in the Title; this will show up in other places once
% we add headers. Instead, do the formatting manually

\title{ACS L108 Lecture Notes: Category Theory, Type Theory, and Logic}
\author{Marcelo Fiore}
\date{Michaelmas Term 2017--18}

%%%% Environment for Labeling Chapters
\newcommand{\lecturedetails}[2]{
    \vspace*{-10mm}\hspace*{7.75mm}
    \parbox{0.7\textwidth} {
        #2\\
        \textit{#1}
    }
    \vspace{10mm}
}

% make sure that chapters are called lectures
\makeatletter
\renewcommand{\@chapapp}{Lecture}
\makeatother

%%%%%

\theoremstyle{definition}
\newtheorem{definition}{Definition}
\newtheorem{lemma}{Lemma}
\newtheorem{theorem}{Theorem}
\newtheorem{proposition}{Proposition}
\newtheorem{exercise}{Exercise}

\newtheorem*{remark}{Remark}
\newtheorem*{example}{Example}

%%%% Stylistic
\setlength\parindent{0pt}
\setlength\parskip{0.7em}

%%%% Category Theory Macros
\newcommand {\cat}{%
    \mathbf%
}
\newcommand {\domain}[1] {%
    \mathrm{dom}(#1)%
}
\newcommand {\codomain}[1] {%
    \mathrm{cod}(#1)%
}
\newcommand {\idarrow}[1][] {%
    \mathbf{1}{#1}%
}
%%%% Macros for specific categories
\newcommand{\catcat}{\cat {Cat}}
\newcommand{\catmon}{\cat {Mon}}
\newcommand{\catposet}{\cat {Poset}}
\newcommand{\catrel}{\cat {Rel}}
\newcommand{\catset}{\cat {Set}}
\newcommand{\catpfn}{\cat {Pfn}}
\newcommand{\catfinset}{\cat {FinSet}}
\newcommand{\catbij}{\cat {Bij}}
\newcommand{\catgroup}{\cat {Groups}}
\newcommand{\catgraph}{\cat {Graphs}}
\newcommand{\catpreo}{\cat {PreO}}
\newcommand{\catpred}{\cat {Pred}}
\newcommand{\catdirgraph}{\cat {DirGraph}}

%%%% Macros for specific functors
\newcommand{\fnclist}{\mathit{List}}
\newcommand{\fncid}[1]{\mathrm{Id}_{#1}}

%%%% Objects and morphism sets
\DeclareMathOperator{\Ob}{\mathit{Obj}}
\DeclareMathOperator{\Arr}{\mathit{Arr}}

\newcommand{\ie}{\emph{i.e.}}
\newcommand{\etc}{\emph{etc.}}
\newcommand{\viz}{\emph{viz.}}

\newcommand{\eqdef}{\stackrel{\text{def}}{=}}
\newcommand{\iffdef}{\stackrel{\text{def}}{\iff}}
\newcommand{\comp}{\circ} % composition
\newcommand{\icomp}{\,} % implicit composition
\newcommand{\iso}{\cong}

\newcommand{\setof}[1]{ \{ #1 \} }
\newcommand{\bigsetof}[1]{ \big\{ #1 \big\} }
\newcommand{\suchthat}{\mid}
\newcommand{\union}{\cup}
\newcommand{\lub}{\bigsqcup} % least upper bound

\newcommand{\nelem}[1]{ \mathbb{ #1 } }
\newcommand{\id}[1]{ \mathrm{id}_{ #1 } }
\newcommand{\Id}[1]{ \mathrm{Id}_{ #1 } }
\newcommand{\idfunc}{ \mathrm{Id} }
\newcommand{\map}[1]{ \mathrm{map}\, #1}
\newcommand{\nats}{\mathbb{N}}
\newcommand{\morpharrow}{\longrightarrow}
\newcommand{\morpharr}[1]{\overset{#1}{\morpharrow}}
\newcommand{\natto}{\Rightarrow}
\newcommand{\natarrow}{\Longrightarrow}
\newcommand{\natarr}[1]{\overset{#1}{\natarrow}}
\newcommand{\lscat}[1]{\mathcal{#1}} % locally small category
\newcommand{\scat}[1]{\mathbb{#1}} % small category
\newcommand{\univpair}[1]{\langle #1 \rangle}
\newcommand{\bigunivpair}[1]{\big\langle #1 \big\rangle}

\newcommand{\seq}[1]{\langle #1 \rangle}

\newcommand{\yon}{\mathrm{y}}

\newcommand{\verteq}{\rotatebox{90}{$\,=$}}
\newcommand{\equalto}[2]{\underset{\displaystyle\overset{\mkern4mu\verteq}{#2}}{#1}}
\newcommand{\mapsfrom}{\reflectbox{$\;\mapsto\;$}}

\newcommand{\phold}{\text{--}}
\newcommand{\Phold}{\text{=}}

\newcommand{\op}{\mathrm{op}}
\newcommand{\dom}{\mathrm{dom}}

% Double-lined fractions, notation for bijective correspondence
% https://tex.stackexchange.com/questions/56003/fraction-with-doubled-line
\newcommand{\Efrac}[2]{%
  \mathchoice
    {\ooalign{%
      $\genfrac{}{}{2.0pt}0{#1}{#2}$\cr%
      $\color{white}\genfrac{}{}{1.0pt}0{\phantom{#1}}{\phantom{#2}}$}}%
    {\ooalign{%
      $\genfrac{}{}{2.0pt}1{#1}{#2}$\cr%
      $\color{white}\genfrac{}{}{1.0pt}1{\phantom{#1}}{\phantom{#2}}$}}%
    {\ooalign{%
      $\genfrac{}{}{2.0pt}2{#1}{#2}$\cr%
      $\color{white}\genfrac{}{}{1.0pt}2{\phantom{#1}}{\phantom{#2}}$}}%
    {\ooalign{%
      $\genfrac{}{}{2.0pt}3{#1}{#2}$\cr%
      $\color{white}\genfrac{}{}{1.0pt}3{\phantom{#1}}{\phantom{#2}}$}}%
}

\newcommand{\efrac}[2]{%
  \mathchoice
    {\ooalign{%
      $\genfrac{}{}{2.0pt}0{\hphantom{#1}}{\hphantom{#2}}$\cr%
      $\color{white}\genfrac{}{}{1.0pt}0{\color{black}#1}{\color{black}#2}$}}%
    {\ooalign{%
      $\genfrac{}{}{2.0pt}1{\hphantom{#1}}{\hphantom{#2}}$\cr%
      $\color{white}\genfrac{}{}{1.0pt}1{\color{black}#1}{\color{black}#2}$}}%
    {\ooalign{%
      $\genfrac{}{}{2.0pt}2{\hphantom{#1}}{\hphantom{#2}}$\cr%
      $\color{white}\genfrac{}{}{1.0pt}2{\color{black}#1}{\color{black}#2}$}}%
    {\ooalign{%
      $\genfrac{}{}{2.0pt}3{\hphantom{#1}}{\hphantom{#2}}$\cr%
      $\color{white}\genfrac{}{}{1.0pt}3{\color{black}#1}{\color{black}#2}$}}%
}

\begin{document}

\begin{center} {\LARGE \sc
Category Theory, Type Theory, and Logic\\
  Lecture Notes\\[4mm]}
  \Large Marcelo Fiore
\end{center}


\thispagestyle{plain}
\begin{multicols}{3}[\section*{Contributors}]
Nathanael Alcock\\
Sebastian Borgeaud\\
Pascal Bose\\
Vishal Chakraborty\\
James Clarke\\
Patrick Fernandes\\
Brad Hardy\\
Andrej Ivašković\\
Razvan Kusztos\\
Stella Lau\\
Nándor Licker\\
Dhruv Makwana\\
Thomas Parks\\
Oliver Richardson\\
Shaun Steenkamp\\
Dima Szamozvancev\\
Zongzhe Yuan\\
\end{multicols}
\clearpage


\tableofcontents

\newpage

\include{tex/lecture1}

\include{tex/lecture2}

\include{tex/lecture3}

\include{tex/lecture4}

\include{tex/lecture5}

\include{tex/lecture6}

\include{tex/lecture7}

\include{tex/lecture8}

\include{tex/lecture9}

\include{tex/lecture10}

\include{tex/lecture11}

\include{tex/lecture12}

\chapter{Adjoint Functors}
\lecturedetails{16 November 2017}{M Fiore, R Kusztos}

\section{More about subobjects classifiers}

Recall the definition of the \underline{Sub} functor: 

\begin{align*}
  \underline{Sub}: \lscat{C}^{op} &\longrightarrow \textbf{Set} \\
  X &\longmapsto \underline{Sub}(X) = \bigl\{ [m]_{\approx} | 
    \textrm{m is mono into X} \bigl\} \\ 
\end{align*}

\begin{center}
  \begin{tikzcd}
    f^{*}(P)
      \arrow[phantom, "\lrcorner", pos=0]{dr} 
      \arrow[r] 
      \arrow[d] & 
    X 
      \arrow[d] \\
    P
      \arrow[r] & 
    Y
  \end{tikzcd}
\end{center}

What is a representation for this?
\begin{definition}[Subobject classifiers]
  To give a natural isomorphism, 
  $\lscat{C}(\_,\Omega) \overset{\iso}{\Longrightarrow} \underline{Sub}(\_) $ 
  is equivalent to giving an 
  $\Omega \in \lscat{C}$ 
  and a 
  $ \Bigl[t \underset{\Omega}{\overset{T}{\downarrow}} \Bigl] 
    \in \underline{Sub}(\Omega)$
  such that for all 
  $X \in \lscat{C}^{\Omega}$
  there exists a bijective correspondence: 

  \begin{align*}
    \forall 
    \Bigl[m \underset{X}{\overset{P}{\downarrow}} \Bigl] 
    \in \underline{Sub}(X) \quad
    \exists! 
    X \overset{\chi_{[m]}}{\longrightarrow} \Omega \quad 
    \textrm{such that} \quad
    \chi_{[m]}^{*}[t] = [m]
  \end{align*}
\end{definition}

\begin{definition}[Subobject classifiers]
  \label{subobject_classifier_def}
  The previous definition is equivalent to simply giving 
  $\Omega \in \lscat{C}$
  and 
  $t \underset{\Omega}{\overset{T}{\downarrow}} \in \lscat{C}$
  such that 
  \begin{align*}
    \forall 
    m \underset{X}{\overset{P}{\downarrow}}
    \in \underline{Sub}(X) \quad
    \exists! 
    X \overset{\chi_{m}}{\longrightarrow} \Omega
  \end{align*}  
  \begin{center}
    \begin{tikzcd}
    P 
      \arrow[phantom, "\lrcorner", pos=0]{dr} 
      \arrow[r, "u"] 
      \arrow[d, "m"] & 
    T 
      \arrow[d, "t"] \\
    X
      \arrow[r, "\chi_m"] & 
    \Omega
    \end{tikzcd}
    \end{center}
  for some $u$
\end{definition}

\begin{example}[in $\textbf{Set}$]
\begin{align*}
  1 \overset{t}{\longrightarrow} \{t, f\} = \Omega
\end{align*}
\begin{align*}
  Sub(X) = \mathcal{P}(X)
\end{align*}

\begin{center}
  \begin{tikzcd}
    P 
      \arrow[phantom, "\lrcorner", pos=0]{dr} 
      \arrow[r, "u"] 
      \arrow[d, "m"] & 
    1
      \arrow[d, "t"] \\
    X
      \arrow[r, "\exists!\varphi"] & 
    \{t, f\}
    \end{tikzcd}
  \end{center}

Where $\varphi$ is such that $\varphi^{-1}\{t\} = P$ \\
Since we are in set, $\varphi$ is the characteristic function of X

\begin{equation*}
 \varphi = \chi_m(x) = 
  \begin{cases}
    t & x \in P \\
    f & x \notin P 
  \end{cases}
\end{equation*}

\end{example}

\begin{exercise}
  In the subobject classifier definition 
  (Definition \ref{subobject_classifier_def}), $T$ is necessarily a
terminal object
\end{exercise}

\begin{definition}[Toposes]
  A topos is a category with: 
  \begin{itemize}
    \item all finite limits 
    \item exponentials
    \item subobject classifier $\Omega$
  \end{itemize}

Examples of toposes are: 
  \textbf{Set}, 
  $\textbf{Set}^{\mathbb{C}^{op}}$, 
  \textbf{DirGraph}

\end{definition}

\begin{remark}
  Toposes are models of Higher Intuitionistic Logic.
\end{remark}

\begin{exercise}
  What is $\Omega$ in \textbf{DirGraph}?

  For some graph 
  $G = (N, E)$, 
  Sub(G) = the set of all its subgraphs, each given by a subset of nodes,
  $S_n \subseteq N$
  and a subset of edges 
  $E_n \subseteq E$ 
  between nodes in $S_n$.

  \begin{center}
    \begin{tikzcd}
      \bullet 
        \arrow[rr, bend left] & {} \arrow[dd, hook'] & 
        \bullet \arrow[loop, looseness=4] &  &  &  & 
        \bullet \arrow[dd, hook'] \arrow[loop, looseness=4] \\
       &  &  &  &  &  &  \\
       & {} &  &  &  &  & {} \\
      \bullet \arrow[rr, bend left] 
      \arrow[rd, bend right] &  & 
      \bullet
        & {} 
      \arrow[rr] &  & {} & 
      \textcolor{green}{\bullet} 
      \arrow[loop left, green]
      \arrow[loop right, red]
      \arrow[bend left=20, blue]{d} \\
       & \bullet \arrow[ru, bend right] \arrow[lu, bend right] &  &  &  &  & 
       \textcolor{red}{\bullet} 
       \arrow[loop right, red]
       \arrow[bend left=20, blue]{u}
    \end{tikzcd}
  \end{center}
\end{exercise}

\begin{exercise}
  What is $\Omega$ in $\textbf{Set}^{(0 \to 1)}$, also known as the 
Sierpinski topos.
\end{exercise}

\section{Adjoint Functors}

\begin{example}
  In the first lecture we introduced the free functor which maps a set to 
the free monoid:

  \begin{align*}
    \textbf{Set} &\overset{F}{\longrightarrow} \textbf{Mon} \\
    S &\longmapsto (S^*, t, \circ)
  \end{align*}
  Similarly, we can introduce the forgetful functor U, which gives the set 
underlying a monoid.

\begin{center}
  \begin{tikzcd}[sep=3em]
    \textbf{Mon}
      \ar["U"{right}]{d} 
    \\
    \textbf{Set}
      \ar[bend left=40, "F"{left}]{u}
  \end{tikzcd}\\[1mm]
\end{center}

Here, we observe that:

\begin{align*}
  \textbf{Mon}(F(S), \underline{M}) &\cong_{nat} 
    \textbf{Set}(S, U(\underline(M))) \\
    \shortintertext{We write that:} \\
    F &\dashv U \\
    \intertext{To say that F and U are adjoints; F is called the left adjoint
    whereas U is called the right adjoint}
  \end{align*}
\end{example}

\begin{example}[Colimits]
  \begin{align*}
    \lscat{C} &\underset{\Delta}{\longrightarrow} \lscat{C}^\mathbb{G} \\ 
    X &\longmapsto \Delta(X): \mathbb{C} \to \lscat{G} \quad 
    \textrm{which maps} \quad
    e \overset{m}{\underset{n}{\downarrow}} \longmapsto
    \overset{x}{\underset{x}{\downarrow}}id
  \end{align*} 
That is, 
  \begin{align*}
    \textrm{For} \quad D \in \lscat{C}^\mathbb{G},  \quad
    &\varphi: D \Rightarrow \Delta(X) \\
    &\equiv \textrm{cocone with vertex x for diagram D} \\
    \lscat{C}(\underline{colim}D, X) &\cong 
      \lscat{C}^{\mathbb{G}}(D, \Delta(X)) \\ 
    \underline{colim} &\dashv \Delta \\ 
    \shortintertext{Dually,} \\ 
    \lscat{C}(X, \underline{lim}D) &\cong
      \lscat{C}^{\mathbb{G}}(\Delta(X), D) \\
    \Delta &\dashv \underline{lim}  
  \end{align*}

\end{example}

\begin{definition}[Adjoint Pair]
An adojoint pair of functors, $F \dashv G: \lscat{D} \to \lscat{C}$ consists
of: 
 \begin{center}
  \begin{tikzcd}
    \lscat{D} 
    \ar[bend left=20, "G"]{r} &
   \lscat{C} 
    \ar[bend left=20, "F"]{l}
  \end{tikzcd}
\end{center}
together with a natural isomorphism:
\begin{align*}
\lscat{D}(F C, D) &\cong 
  \lscat{C}(C, G D) \\ 
\end{align*}

Equivalently, to give a left adjoint $F$ to a functor 
$G: \lscat{D} \to \lscat{G}$
is to give for all $X \in \lscat{C}$
and object $F(x) \in \lscat{D}$
together with a map $X \underset{\eta_X}{\rightarrow} G(F(X)$
in $\lscat{C} $
such that: 
\begin{center}
    \begin{tikzcd}[ampersand replacement=\&]
    X \arrow[r, "\eta_X"]
    \arrow[dr, "\forall f"{left}] \&
    G(F(X)) \arrow[d, "G(f^\#)"] \& 
    F(X) \arrow[d, mapsto, "\exists! f^\#"] \\
    {} \& G(Y) \& Y
  \end{tikzcd} 
\end{center}
\end{definition}
\begin{proposition}
  Left adjoints preserve colimits and dually, right adjoints preserve limits.
\end{proposition}

\begin{example}
A functor $F: \lscat{C} \to \lscat{D}$ preserves sums if:

  \begin{center}
    \begin{tabular}[t]{lc}
    $\forall$ sums in $\lscat{C}$
    \begin{tikzcd}
      {} & A + B & {} \\
      A \ar["\iota_1"]{ru} & & B \ar["\iota_2"above]{lu} \\
    \end{tikzcd}
    ,
    \begin{tikzcd}
      {} & F(A + B) & {} \\
      F(A) \ar["F(\iota_1)"]{ru} & & F(B) \ar["F(\iota_2)"above]{lu} \\
    \end{tikzcd}
    is a sum in $\lscat{D}$.
    \end{tabular}
  \end{center}

If $\lscat{D}$ has sums, 

\begin{center}
  \begin{tikzcd}
    & F(A) + F(B) \ar["\cong", dotted]{d} & \\ 
    & F(A + B) & \\
    F(A) \ar["F(\iota_1)"]{ru} 
         \ar["\iota^{F(A), F(B)}_1", bend left=40]{ruu} & & 
    F(B) \ar["F(\iota_2)"]{lu}
         \ar["\iota^{F(A), F(B)}_2", bend right=40]{luu} \\
    

  \end{tikzcd}
\end{center} 

We need to show that $F(A) + F(B) \cong F(A + B)$. We have previously shown
that this is equivalent to showing that: 

\[ 
  \Efrac
    {F(A) + F(B) \to z}
    {F(A + B) \to z}
\]

Intuitively,

\LARGE\[
  \Efrac
  {
  \Efrac
    {F(A) + F(B) \to z} 
    { \Efrac 
      {F(A) \to z}  
      {A \to G z}
      \quad 
      \Efrac 
      {F(B) \to z}
      {B \to G z}
    }
  }
  {
    \Efrac 
    {A + B \to G z}
    {F(A + B) \to z}
  }
\]

\end{example}

\begin{example}
  Exponentials are adjoints.
  \begin{align*}
    \_ \times A &\dashv A \Rightarrow (\_) \\
    \lscat{C}(X (\times A), B) &\cong \lscat{C}(X, (A \Rightarrow) B)
  \end{align*}
\end{example} 


\end{document}
